\documentclass[zihao=-4]{ctexart}
\usepackage[normalem]{ulem}
\useunder{\uline}{\ul}{}
%********************导言区宏包引入********************
\usepackage{xeCJK}
\usepackage{amssymb}
\usepackage{amsmath}
\usepackage{listings} %代码
\usepackage{graphicx}
\usepackage{multicol} %回车换段
\usepackage{xcolor}
\usepackage{geometry} %页面设置
\usepackage{fontspec}
\usepackage{setspace}
\usepackage{times}
\usepackage{fancyhdr} %页眉页脚
\pagestyle{fancy}
\usepackage{float} %表格位置
\usepackage{titlesec} %设置
\usepackage{titletoc}
\usepackage{ctex}
\usepackage{gbt7714}    %控制参考文献格式为国标
\usepackage{multirow}
\usepackage{booktabs}   %表格相关
\usepackage{setspace}   %设置行距
\usepackage{caption} %caption
\usepackage{subcaption} %子图的caption
\usepackage{changepage} %左右缩进


\graphicspath{ {pictures/} }
\let\algorithm\relax
\let\endalgorithm\relax
\usepackage[ruled,vlined]{algorithm2e}%[ruled,vlined]{
\usepackage{algpseudocode}
\renewcommand{\algorithmicrequire}{\textbf{Input:}} 
\renewcommand{\algorithmicensure}{\textbf{Output:}}
%\renewcommand\thepage{\zihao{-5} ~\arabic{page}~}%页码字号

%定义两个arg
\DeclareMathOperator*{\argmax}{arg\,max}
\DeclareMathOperator*{\argmin}{arg\,min}
\DeclareCaptionLabelSeparator{mysep}{\space\space}  %自定义caption格式
\captionsetup[figure]{font={small}, labelfont=bf, labelsep=mysep, textfont=bf}   %图片caption格式
\captionsetup[table]{font={small}, labelfont=bf, labelsep=mysep, textfont={bf}}   %表格caption格式
\bibliographystyle{gbt7714-numerical} %修改了title斜体内容

%********************导言区宏包引入********************
%********************第三方字体引入********************
%\setCJKmainfont[Path=fonts/,BoldFont=simhei.ttf,ItalicFont=simkai.ttf,SlantedFont=simfang.ttf]{simsun.ttc}
%中文字体涵盖黑体、宋体、楷体、仿宋
\setmainfont[Path=fonts/, 
BoldFont = times-new-roman-bold.ttf,
ItalicFont = times-new-roman-italic.ttf,
BoldItalicFont = times-new-roman-bold-italic.ttf
]{times-new-roman.ttf}
\setmonofont[Path=fonts/]{Courier New.ttf}
\setCJKfamilyfont{hwzs}[Path=fonts/]{STKzhongsong.ttf}%使用STZhogsong华文中宋字体
\newcommand{\zhongsong}{\CJKfamily{hwzs}}
\setCJKfamilyfont{hwxw}[Path=fonts/]{STKxinwei.ttf} % XSP 2023/3/3:
\newcommand{\xinwei}{\CJKfamily{hwxw}}              %  使用STZxinwei华文新魏字体.

%********************第三方字体引入********************

%********************代码段设置********************
% by yjw
% \texttt 用于行内代码,等宽字体,不知道让不让用
\usepackage{listings} % 用于代码排版
\usepackage{xcolor} % 用于代码高亮

% 定义代码块样式
\lstdefinestyle{codeblock}{
    backgroundcolor=\color{white},    % 背景颜色
    basicstyle=\ttfamily\small,       % 设置字体样式
    breakatwhitespace=false,          % 是否只在空白处自动断行
    breaklines=true,                  % 自动断行
    captionpos=b,                     % 标题位置
    commentstyle=\color{gray},       % 注释风格
    frame=single,                     % 单框
    keepspaces=true,                  % 保持空格
    keywordstyle=\color{blue},        % 关键字风格
    numbers=left,                     % 行号位置
    numbersep=8pt,                    % 行号与代码的距离
    numberstyle=\tiny\color{gray},    % 行号样式
    rulecolor=\color{black},          % 框架颜色
    showspaces=false,                 % 不显示空格
    showstringspaces=false,           % 字符串中不显示空格
    showtabs=false,                   % 不显示制表符
    stepnumber=1,                     % 步长
    stringstyle=\color{orange},       % 字符串风格
    tabsize=2,                        % 制表符占用空格数
    xleftmargin=\parindent,           % 左边距
    framexleftmargin=12pt,             % 框架左边距,增加些许间距以包含行号
}
\renewcommand{\lstlistingname}{伪代码}
\lstset{style=codeblock}

\usepackage{hyperref} %超链接支持
%********************代码段设置********************

%********************中文字号设置********************
%\newcommand{\chuhao}{\fontsize{42pt}{\baselineskip}\selectfont}
\newcommand{\chuhao}{\fontsize{42pt}{0}}
\newcommand{\xiaochu}{\fontsize{36pt}{0}}
\newcommand{\yihao}{\fontsize{28pt}{0}}
\newcommand{\erhao}{\fontsize{21pt}{0}}
\newcommand{\xiaoer}{\fontsize{18pt}{0}}
\newcommand{\sanhao}{\fontsize{16pt}{0}}
\newcommand{\sihao}{\fontsize{14pt}{0}}
\newcommand{\xiaosi}{\fontsize{12pt}{0}}
\newcommand{\wuhao}{\fontsize{10.5pt}{0}}
\newcommand{\xiaowu}{\fontsize{9pt}{0}}
\newcommand{\liuhao}{\fontsize{8pt}{0}}
\newcommand{\qihao}{\fontsize{5.25pt}{0}}
%********************中文字号设置********************


%********************页边距设置********************
\geometry{left=3cm,right=2cm,top=2.5cm,bottom=2.5cm}
\geometry{a4paper} % xsp 2023/3/7: 调整纸张大小为A4
%********************页边距设置********************

%********************段间距设置********************
\newcommand{\setParDis}{\setlength {\parskip} {0pt} }
%请在每部分使用这个
%********************段间距设置********************

%********************子文件导入区域****************
% 用于仅编译部分导入文件
% \includeonly{subtexs/ros_background, }

\begin{document}
%********************页眉页脚设置********************
\lhead{}%设置左页眉为空
\rhead{}%设置左页眉为空
\setlength{\headwidth}{\textwidth}% 2023/3/3 XSP: 页眉长度适应文本
%********************页眉页脚设置********************


%********************标题格式设置********************

%\setcounter{secnumdepth}{0}%该命令取消了章标题前数字label

\CTEXsetup[name={,、},number={\chinese{section}}]{section}
\CTEXsetup[name={(,)},number={\chinese{subsection}}]{subsection}
\CTEXsetup[name={,.},number={\arabic{subsubsection}}]{subsubsection}% 不加会导致目录格式错误
% 设置subsubsection等格式
% \titleformat{\section}[block]{\sanhao\bfseries\centering}{\chinese{section}、}{0pt}{}[]
% \titleformat{\subsection}[block]{\sihao\bfseries}{(\chinese{subsection})}{0pt}{}[]
% \titleformat{\subsubsection}[block]{\xiaosi\bfseries}{\arabic{subsubsection}、}{0pt}{}[]
\titleformat{\section}[block]{\sanhao\heiti\centering}{\chinese{section}、}{0pt}{}[]    % XSP 2023/3/3:
\titleformat{\subsection}[block]{\sihao\heiti}{(\chinese{subsection})}{0pt}{}[]       %   将正文标题字体由加粗
\titleformat{\subsubsection}[block]{\xiaosi\heiti}{\arabic{subsubsection}.}{0pt}{}[]   % 修改为黑体。
\titlespacing{\section}{0pt}{25pt}{12pt}
\titlespacing{\subsection}{0pt}{7pt}{7pt}
\titlespacing{\subsubsection}{0pt}{5pt}{4pt}

\titlecontents{section}[1.6em]{\addvspace{2pt}\filright}
{\contentspush{\thecontentslabel\hspace{0.8em}}}
{}{\titlerule*[8pt]{.}\contentspage}

\titlecontents{subsection}[3.2em]{\addvspace{2pt}\filright}
{\contentspush{\thecontentslabel\hspace{0.8em}}}
{}{\titlerule*[8pt]{.}\contentspage}

\titlecontents{subsubsection}[6.4em]{\addvspace{2pt}\filright}
{\contentspush{\thecontentslabel\hspace{0.8em}}}
{}{\titlerule*[8pt]{.}\contentspage}
%********************标题格式设置********************

%\setcounter{section}{-3}  %标题计数器
%\stepcounter{section}

%*******************行间距段前段后*******************
\linespread{1.8}
%行间距为实际行间距乘以1.2,如此处实际为1.5倍行距
\setlength{\parskip}{0.5\baselineskip}
%*******************行间距段前段后*******************



%********************封面部分********************
%
%     论文题目:应准确、鲜明、简洁,能概括整个论文中最主要和最重要的内容。
% 题目不超过20个中文字,若语意未尽,可用副标题补充说明。副标题应处于从属
% 地位,一般可在题目的下一行用破折号“——”引出。论文题目应避免使用不常用缩
% 略词、首字母缩写字、字符、代号和公式等。
%
\def\Fengru{第三十四届“冯如杯”竞赛主赛道}
\leftline{\includegraphics[scale=1]{pictures/xiaohui.png}} % XSP 2023/3/3: 取消校徽段首缩进
%格式控制部分
% \par \  
% \par \
% \par \
\vspace{32pt}
\begin{center}
\includegraphics[height=2.25cm, width=12.78cm, scale=1]{pictures/xiaoming.png}
\end{center}
%格式控制部分
\vspace{12pt}

\begin{spacing}{3}
    % \erhao
    \begin{center}
      {
        \fontsize{22pt}{3}\selectfont
        \zhongsong{\Fengru 项目论文Latex模板} %黑体这样调用,其余字体同理
      } 
        % \zhongsong{“冯如杯”竞赛主赛道项目是什么}
    \end{center}
    \rightline{\xinwei\sanhao{——基于 Latex 的论文模板}} % XSP 2023/3/3: 副标题二号华文新魏居右
\end{spacing}
%格式控制部分
% \par \ 
% \par \
\par \ 
\par \
\par \ 
\par \
% \begin{center}
%     \sihao
%     \textbf{学院:计算机学院}
%     \par \ 
%     \textbf{本模板原作者:Someday}
% \end{center}

%格式控制部分
\par \ 
\begin{center}
\sanhao
\centerline{\heiti{}}%封面年月去掉
\end{center}

\pagenumbering{gobble} %封面无页码
%\thispagestyle{empty}


\renewcommand{\headrulewidth}{0pt}%没有页眉装饰线
\clearpage
\pagenumbering{roman} %摘要目录页小写罗马

\xiaosi
\section*{摘要}
\begin{spacing}{1.5}
  \setParDis %设置段间距为 0
% 500z字? 目的, 意义, 研究方法, 成果, 结论
商用机器人、自动驾驶等领域的快速发展,带动了相关软件工具的进步。其中,机器人操作
系统(ROS2),以其开源、跨平台、高效的软件框架及丰富的工具库而深受开发者喜爱,在
学界、商业应用中得到广泛应用。与此同时,机器人操作系统也存在开源软件共有的安全漏
洞多、不稳定等问题,制约着其进一步发展,也威胁着相关应用的安全。

机器人操作系统相关程序亟需一种高效的测试方法,面对这一场景,我们选择将自动模糊测
试技术(Fuzzing)引入对机器人操作系统的测试。通过随机变异测试输入,并加入反馈引
导,模糊测试不仅提高了测试的自动化程度,还能有效地探索软件异常路径,揭露隐蔽漏
洞,显著改善了传统软件测试方法的繁琐性和低效性。

为了提高测试效率,我们的模糊测试对机器人系统进行针对性优化。针对通信网络环境的不
稳定性,本框架选择将机器人收到的通信消息等多维度输入进行随机化变异;针对机器人多
进程多线程的软件特点,本框架通过程序插桩延迟来提高数据竞争概率;针对机器人程序弱
状态特点,本框架通过流量分析等黑盒手段,来辅助推测程序内部状态。

本作品设计的测试框架短期内共发掘20个ROS2相关软件的内存漏洞、并发漏洞,其中11个得
到开发者确认修复,其中有4个重要漏洞被美国通用漏洞披露(CVE)收录,7个通过中国国
家漏洞库(CNVD)二级审核,正在等待三级审核。

综上,本作品——针对机器人系统的模糊测试框架在软件测试中有较高实践价值,已为开源社
区贡献多次漏洞修复,并可高效地移植用于测试其他机器人程序。


\end{spacing}
    
\textbf{关键词:}自动模糊测试技术,机器人操作系统,并发漏洞检测,内存安全漏洞检测

\newpage
\section*{\textbf{Abstract}} % XSP 2023/3/8: Abstract 加粗
\begin{spacing}{1.5}
\begin{adjustwidth}{0.42cm}{0.42cm}
  \setParDis %设置段间距为 0

The swift advancement in commercial robotics and autonomous driving has spurred
the development of associated software tools, notably the Robot Operating System
(ROS2). Celebrated for its open-source, cross-platform capabilities, and
extensive toolkit, ROS2 has garnered widespread adoption across academia and
industry. However, its growth is hampered by common open-source software issues
like security vulnerabilities and instability, posing risks to application
safety.

Addressing the need for efficient testing in \textbf{ROS2-related programs}, we
introduced automatic \textbf{fuzz} testing to enhance software reliability. By
employing randomized mutations of test inputs and feedback guidance, fuzz
testing significantly elevates automation and efficacy in exploiting
vulnerabilities, outpacing traditional testing in identifing uncovering software
exception.

Our framework, tailored for robotic systems, addresses communication network
instability by randomizing inputs like received communication messages. It
targets software's multi-process and multi-threaded nature by increasing data
race probabilities via program instrumentation delays. Additionally, to navigate
the challenges posed by robots' weak states, our framework utilizes black-box
methods, such as traffic analysis, to infer internal states. The approach led to
\textbf{the discovery of 20 ROS2 software vulnerabilities, with 11 confirmed and
fixed by developers, including 4 recognized by the US's CVE and 7 awaiting final
review by China's CNVD.}

This work demonstrates the high practical value of our fuzz testing framework
for robotic systems, contributing significantly to the open-source community and
efficiently adaptable for testing various robot programs.

\textbf{Keywords: Fuzz, ROS2, Concurrent Vulnerability, Memory Vulnerability}
\end{adjustwidth}
\end{spacing}



%********************摘要部分********************


%********************目录部分********************
\clearpage
\tableofcontents
\clearpage
%********************目录部分********************



\renewcommand{\headrulewidth}{0.4pt} %恢复页眉装饰线

%********************正文页眉部分********************
%\lhead{} 
\chead{\xiaowu 北京航空航天大学\Fengru 参赛作品} %设置居中页眉
%********************正文页眉部分********************

\pagenumbering{arabic} %正文页码从1开始,用阿拉伯数字
\setcounter{page}{1} 

%%%%%%%%%%%%%%%%%%%%%% 一 作品概述
\section{作品概述}
  \setParDis %设置段间距为 0
\begin{spacing}{1.5} % 行距1.5


\subsection{背景介绍}
\setParDis %设置段间距为 0
%@author: lhy
% \subsection{背景介绍}
%@author: lhy
\subsubsection{ROS系统重要性}
人类生活与机器人密不可分。传统上,机器人在农业和制造业中被广泛用于任务自动化。最
近的发展让机器人更加贴近每个人的日常生活。例如,亚马逊和谷歌正在部署无人配送系统
[22,23],使用无人驾驶飞行器,机器人吸尘器的市场规模年增长率达到23\%[24],以及在
2012年到2018年间,使用机器人进行手术的比例从2\%增加到15\%[25],显示出机器人产业
为适应人类需求而快速增长的态势。同时,现代机器人,如自动驾驶车辆,正在变得更加复
杂,要求集成复杂的子系统,例如感知、感觉、规划和执行。

为了应对开发复杂机器人系统的更高需求,机器人操作系统(ROS)[1]正在获得越来越多的
关注。ROS是一个开源的中间件套件,用于机器人开发,它具有分布式机器人进程的消息传
递机制、硬件抽象、广泛的开发工具(例如,模拟器)和机器人库(例如,路径规划算
法)。使用ROS,开发者可以加速开发过程,无需重新发明轮子,而是可以专注于他们机器
人的核心功能。凭借其多语言和多平台支持的理念,ROS正在成为机器人编程中的实际标
准;它已被广泛采用于工业[26,27]、军事[28,29]、研究机构以及个人,预计到2024年将为
55\%的商业机器人提供动力[30]。近些年ROS系统相关论文数量与Wiki用户数量如图1所示:

\begin{figure}[H]
  \centering
  \includegraphics[width=0.8\textwidth]{ros_papers.png}
  \caption{ROS系统相关论文数量与Wiki用户数量}
\end{figure}

机器人操作系统(ROS)是一个开源平台,它为用C++和Python等不同语言实现机器人软件提
供了许多实用的工具和库。超过740家商业公司正在使用ROS[2]来开发数百种实际应用的机
器人[3],这表明了其在机器人开发中的流行度。然而,开发可靠的ROS程序具有挑战性,因
为机器人可以在复杂的物理环境中工作,并可能遇到各种异常情况(如无效的配置参数和异
常的传感器信息)。如果ROS程序不够可靠,它们的错误可能在与人类和其他机器人交互时
导致危险事故。

ROS本身的缺陷或者开发者在应用中常用的ROS使用方式可能会影响依赖于ROS的广泛机器人
系统,并对许多用户的安全和保障造成严重破坏。例如,ROS版本1没有认证的概念,允许网
络上的任何实体完全访问任何机器人系统,窃听内部消息,甚至在知道IP地址和端口的情况
下劫持执行。实际上,在互联网上扫描默认的ROS端口两个月后[31],发现部署在28个国家
的超过100个ROS系统,这些系统完全暴露于此类攻击之下。ROS的最新版本(即ROS 2)在设
计和实现时考虑了这些问题,并邀请了更广泛的公众对安全性进行审查。不幸的是,现有的
工作和解决方案要么专注于关于认证和授权方法的网络安全方面[32,33,34,35],要么专注
于回归测试[36],使得ROS社区在寻找影响机器人系统的健壮性和正确性的未知错误方面缺
乏一种系统的测试方法。
\subsubsection{自动驾驶兴起与ROS}
自动驾驶技术的兴起标志着交通运输领域的一次重大革命,其不仅预示着交通安全、效率和
环境影响的根本改进,而且还预示着个人出行和货物运输方式的深刻变革,而这一过程在很
大程度上得益于机器人操作系统(Robot Operating System,ROS)的发展和应用。近年
来,由于计算能力的显著提升、大数据技术的发展、人工智能算法的进步以及传感器技术的
优化,自动驾驶汽车从理论研究和小规模试验逐步走向了公路测试和商业化应用的初步阶
段。

根据国际汽车工程师学会(SAE)的定义[1],自动驾驶分为六个级别,从0级(无自动化)到5
级(完全自动化)。目前,多数公开测试和部分商业化应用的自动驾驶汽车处于3级(有条
件自动化)到4级(高度自动化)。尽管完全自动化(5级)的汽车尚未广泛部署,但多个技
术开发者和制造商已经在进行相关的研发工作,并在不同国家和地区开展了路试。

随着自动驾驶技术的进步,对处理大量传感器数据、实现复杂决策逻辑和执行精确控制的需
求不断增加,ROS的可扩展性和灵活性在此过程中显示出其不可或缺的价值。例如,ROS的消
息传递系统支持多种编程语言,允许异构系统高效集成。此外,其庞大的开源社区贡献了大
量的软件包,覆盖从3D视觉到路径规划等各种功能,极大地丰富了自动驾驶汽车的研发资源
库。

经济学人智库的一份报告预测[2],到2035年,全球自动驾驶汽车的数量将达到近7500万
辆。而麦肯锡公司的一项研究则估计[3],自动驾驶技术的全面部署将使交通事故造成的死
亡人数减少90\%,同时还将大幅度提升道路运输效率和减少碳排放。图2展示了PRECEDENCE
RESEARCH机构预测的未来自动驾驶市场规模:

\begin{figure}[H]
  \centering
  \includegraphics[width=0.8\textwidth]{market_size.png}
  \caption{自动驾驶市场规模}
  \label{fig:my_label}
\end{figure}

尽管自动驾驶技术的前景被广泛看好,但其商业化应用和普及还面临多重挑战,包括技术完
善度、法律法规、道德伦理、消费者接受度以及基础设施配套等。例如,自动驾驶汽车在复
杂的城市交通环境中的表现,以及在极端天气条件下的可靠性,仍然是技术研发中的重要挑
战。

未来,随着相关技术的进一步成熟和相关政策的完善,自动驾驶汽车有望在更广泛的领域得
到应用,从而实现对交通系统的根本性改造和优化。这将不仅影响交通运输本身,还将对城
市规划、能源消费、环境保护以及人们的生活方式产生深远影响。

\subsubsection{自动驾驶产生漏洞的危害}
尽管自动驾驶汽车(AVs)承诺通过减少人为错误来增加道路安全,它们固有的技术漏洞却
可能导致严重的安全后果。据国家公路交通安全管理局(NHTSA)估计,94\%的严重交通事
故是由人为因素引起的,而自动驾驶技术的开发者们宣称,通过消除这一因素,可以极大地
提高道路安全性[4]。然而,这种安全性的提升假设基于自动驾驶系统能够无缺陷地执行其
预定任务,这在实践中往往难以实现。

自动驾驶汽车的技术漏洞主要分为两类:软件漏洞和硬件故障。软件漏洞包括但不限于算法
错误、安全漏洞以及对异常情况的处理不足。硬件故障可能涉及传感器故障、执行机构的故
障或其他关键组件的损坏。这些技术漏洞不仅可能导致单一车辆的操作失败,还可能对整个
交通系统产生连锁反应,特别是在高度自动化和互联的交通环境中。

例如,2018年3月,在美国亚利桑那州发生了一起致命的自动驾驶汽车事故,一名行人在穿
越街道时被一辆自动模式下的Uber汽车撞击,导致死亡。初步调查显示,该事故部分原因是
系统无法正确识别并响应行人[5]。这一事件凸显了即使在自动驾驶技术取得显著进步的情
况下,技术漏洞仍然可能导致灾难性后果。

此外,安全研究人员已经证明,自动驾驶汽车系统的安全漏洞可以被黑客利用,进而控制车
辆的行驶方向或速度,造成严重安全威胁[6]。例如,通过篡改车辆的环境感知系统接收到
的数据,攻击者可以使汽车忽略停车标志或其他关键交通信号,导致交通事故。

综上所述,尽管自动驾驶技术的发展被寄予厚望,能够在根本上改变我们的交通系统,降低
事故率,提高交通效率,但技术漏洞的存在表明,要实现这些目标,还需要克服重大的技术
和安全挑战。这要求汽车制造商、技术开发者、立法者和监管机构合作,不断提高自动驾驶
系统的安全性和可靠性,同时为可能出现的技术漏洞和安全威胁制定全面的应对策略。



\subsection{研究现状}
\setParDis %设置段间距为 0
\subsubsection{ROS常见测试方案}
%@author: yjw
%TODO: 审核,文献,表格格式
近期有几种$^{[2, 6, 7, 8, 9, 10]}$应用于ROS2程序的新型测试方法,它们不同于由开发
者手动构造测试样例的单元测试,而尝试自动生成ROS程序输入,并借助分析检测工具
(Sanitizer)自动监控程序内存或语义错误。它们大幅提高了针对机器人系统的测试效
率,表\ref{tab:fuzzers}将这些方法进行了比较(包括本作品),可以看出这些方法在测
试ROS程序时仍存在三个主要限制:

\textbf{限制一:}测试用例生成效率低。大部分测试方法从单一维度(即用户命令或传感
器信息)生成输入,并且生成的测试样例随机性过强,无法识别出可能高效探索程序边界的
变异维度,导致许多生成的输入无助于增加测试覆盖率。

\textbf{限制二:}程序反馈效果不佳。SMACH-Fuzz和ASTAA随机生成输入,而没有反馈和指
导;Ros2-fuzz和ROZZ参考AFL工具$^{[11]}$使用代码覆盖率作为程序反馈,但代码覆盖率
对多进程运行情况不敏感,且忽略了不同执行路径;Phys-Fuzz使用场景危险程度评分来指
导进一步生成场景,对于检测ROS2自动导航程序的避障能力效果较好,但也仅限制在了此场
景下;RoboFuzz使用语义反馈来量化机器人执行上下文的正确性,但是这种特殊语义的设计
编写需要很强专业知识。

\textbf{限制三:}自动化水平低。大部分方法都需要大量领域特定知识和手动努力来配置
输入生成规则(ASTAA和RoboFuzz),编写有关机器人行为的规格(SMACH-Fuzz和
RoboFuzz),检查执行日志(SMACH-Fuzz和Phy-Fuzz)等,这导致测试效率和便捷程度对比
传统单元测试并没有很大提高。

\begin{table}[H]
\small
\centering
\caption{最新针对ROS系统的测试技术对比}
\begin{tabular}{lccccc}
\hline
\textbf{技术名} & \textbf{输入维度}  & \textbf{反馈}& \textbf{自动化程度} & \textbf{漏洞类型} \\ \hline
SMACH-Fuzz $^{[6]}$ & 单一 & 无 & 差 & 行为错误 \\ 
Phys-Fuzz $^{[7]}$ & 单一  & 环境危险程度  & 差& 碰撞损坏 \\ 
Ros2-fuzz $^{[10]}$ & 单一  & 代码覆盖率   & 较好 & 内存漏洞 \\ 
ASTAA $^{[8]}$ & 单一 & 无 & 差 & 行为错误 \\ 
RoboFuzz $^{[9]}$ & 单一  & 语义反馈 & 较差 & 行为错误 \\ 
ROZZ $^{[2]}$ & 多维度 & 代码覆盖率 & 较好 & 内存漏洞 \\ 
本作品 & 多维度  & 代码覆盖率+流量分析 & 较好& 内存与并发漏洞 \\ \hline
\end{tabular}
\label{tab:fuzzers}
\end{table}

普通单元测试, 代码分析工具 $^{[2]}$

\subsubsection{模糊测试及其应用}
%@author: jfb
强调:模糊测试目前很火,但是没有用到ros上

\subsection{作品概述}
\setParDis %设置段间距为 0
%@author: yjw
% author: yjw
本作品给出针对ROS2系统的自动模糊测试程序,其核心框架如图\ref{pic:off}所示,旨在
提高对ROS2系统相关的自动驾驶项目、机器人项目的漏洞测试的便捷性和高效性,为开发者
提供一个有效测试工具。开发者首先将源码搭配本作品的编译器工具编译为可执行程序,此
过程中本作品将自动使用LLVM架构对源码进行插桩修改,并链接相应检测分析程序(如Address
Sanitizer等第三方监控工具);接着,本作品核心模糊测试驱动器会启动特殊ROS2执行环
境执行开发者的ROS2程序,并将开发者提供的初始测试样例进行变异修改,生成一个测试用
例输入进程序;本作品的监控器(Node Monitor)会对被测程序的代码覆盖率、程序异常、
通信流量、服务状态等信息进行收集分析;分析结果被作为该测试用例种群适应度输入到遗
传算法中,遗传算法会对种群(测试用例池)进行进化和淘汰,选择出更有可能触发程序边
界的测试用例,并指导对测试用例种群进行进一步变异。

\begin{figure}[H]
    \centering
    \includegraphics[width=14cm]{our_fuzz_framework.png}
    \caption{本项目实现的Fuzz框架}
    \label{pic:off}
\end{figure}

通过上述步骤,本作品会不断生成被测ROS2节点程序的高质量测试用例,覆盖更多节点程序
的内部状态分支,以更高效率探索程序潜在漏洞。本作品的创新之处主要有几点:针对ROS2
系统程序特点,不局限于传统测试输入,探索多维度的程序输入;以更高效的办法对ROS2节
点程序进行监控,综合白盒插桩检测与黑盒流量分析等技术;支持检测多种漏洞类型,如内
存安全漏洞与并发漏洞等。

% 终极版本是一个魔改的ROS环境,包括魔改DDS,魔改RCLCPP链接库,Tracing守护进程(流量分析),一个Node可以不运行在其项目整个系统中,如测试Navigation2中的Nav2 AMCL可以不运行整个系统,魔改ROS环境会自动接管其所有接口,进行输入变异



%%%%%%%%%%%%%%%%%%%%%% 二 作品设计与实现
\section{作品设计与实现}
\setParDis %设置段间距为 0

\subsection{技术背景与预备知识}
\setParDis %设置段间距为 0
%@author: jfb
\subsubsection{模糊测试框架}
\subsubsection{漏洞种类及危害}
%@author: wkt
\subsubsection{漏洞种类及危害}
\textbf{释放后使用(UAF)}:

释放后使用漏洞发生在程序试图使用已释放的内存时。通常,内存释放后,操作系统会将该内存标记为可重新使用,但并不会立即清空其内容。如果程序在内存释放后继续访问该内存,就可能导致安全漏洞。该漏洞的介绍如图\ref{fig:UAF}所示。
	\begin{figure}[htbp]
		\centering
		\includegraphics[width=0.5\textwidth]{pictures/UAF.png}
		\caption{释放后使用漏洞图示}
		\label{fig:UAF}
	\end{figure}
攻击者可能利用释放后使用漏洞来执行未经授权的代码,例如通过释放内存但保留对其的引
用,然后在后续代码中使用该引用,从而导致恶意代码执行。同时可能导致程序崩溃或不稳
定,因为操作已释放的内存区域可能导致未定义的行为。

Heartbleed漏洞是一个广为人知的释放后使用漏洞的例子。该漏洞影响了OpenSSL库中的
Heartbeat扩展,攻击者可以发送恶意的Heartbeat请求,从而导致服务器上的内存泄漏和可
能的敏感信息泄露。

\textbf{异常(Exception)}:
	
异常是一种用于处理错误或不正常情况的机制。在程序执行期间,如果发生异常,通常会中
断当前执行流程,并转移到异常处理代码。如果程序未正确处理异常,可能会导致安全漏
洞。

异常可能导致程序跳转到未经测试的代码路径,使得程序执行流程不可预测,从而导致意外
行为或安全漏洞。同时,异常通常包含程序执行的上下文信息,攻击者可能利用这些信息来
进一步渗透系统,进而导致信息泄露。

在2014年的``Shellshock''漏洞中,攻击者利用了Unix/Linux系统中的一个bash shell的异
常处理漏洞。通过在HTTP请求的User-Agent头中注入恶意代码,攻击者能够利用bash的异常
处理漏洞来执行任意命令,从而导致系统被入侵。
	
\textbf{缓冲区溢出(Buffer overflow)}:
	
缓冲区溢出发生在程序试图向一个缓冲区写入超过其分配大小的数据时。这导致数据溢出到
相邻的内存区域,覆盖了那些数据或程序代码。通常,这种溢出可以修改程序的执行流程,
因为溢出数据可能包含特定的指令地址,攻击者可以利用这一点来控制程序的行为。该漏洞
的介绍如图2所示。
	
\begin{figure}[htbp]
  \centering
  \includegraphics[width=0.5\textwidth]{pictures/Buffer Overflow.png}
  \caption{缓冲区溢出漏洞图示}
  \label{fig:BO}
\end{figure}
	
攻击者可能利用缓冲区溢出漏洞来执行恶意代码,例如注入Shellcode并强制程序跳转到
Shellcode的地址,从而获得系统权限。同时,也可能泄露敏感信息,例如通过溢出将重要
的内存区域覆盖为攻击者所控制的数据,从而导致信息泄露。除此之外,因为缓冲区溢出可
能导致程序崩溃或无法正常执行,使其无法提供正常的服务。

著名的``Code Red''蠕虫利用了Microsoft IIS服务器上的缓冲区溢出漏洞。攻击者通过发
送特制的HTTP请求,导致IIS的缓冲区溢出,并在受感染的系统上运行恶意代码。这导致系
统被感染并且在网络上传播蠕虫。
	

\subsubsection{ASan}
\subsubsection{基于覆盖率的遗传算法}
\subsubsection{ROS2系统机制}
\input{subtexs/yjw_ros_background.tex}
%@author: yjw

\subsection{关键设计}
\setParDis %设置段间距为 0
%author: yjw, 2024-4-1
%\subsection{关键设计}
本项目核心目标是将自动模糊测试(Fuzzing)框架迁移到ROS2系统上,并且针对ROS2特性
对测试效率做出改进提升。具体而言,我们试图寻找更多可能的程序输入口,使测试覆盖的
输入可能性尽可能多;尝试构造质量尽可能高的测试用例;寻找更多反馈信息,用于监控被
测程序(即ROS2节点)的内部状态,以更高效地指导对输入的自动变异测试;尝试模拟更多
ROS2执行环境,模拟程序可能遇到的各种实际情况。

\subsubsection{通信环境测试}

如前所述,在机器人操作系统中,节点间通信方式可以基于本地网络套接字、有线通信、进
程内部通信等方式。在实际应用场景中,不能保证网路中通信的稳定性,如图\ref{pic:fmm}所
示,可能出现乱序、重复、丢包、变频等报文序列变化以及报文内容本身的变化。针对这一
观察,我们的ROS Fuzzer尝试对ROS2系统内部通信报文进行类似的变异,以测试ROS2程序的
鲁棒性。

\begin{figure}[h]
    \centering
    \includegraphics[width=10cm]{fuzz_msg_mutation.png}
    \caption{不稳定网路中的报文变化}
    \label{pic:fmm}
\end{figure}

已有测试框架Rozz$^{[2]}$建议在ROS2系统话题通信机制的订阅者和发布者建立恶意中介节
点,它会拦截消息发布者的消息,然后对消息进行变异后再转发给订阅者。本作品认为该方
案适用性有限,首先完全拦截消息难度大,尤其是对于节点内部硬编码的话题;其次ROS2抽
象层提供的QoS机制和时间戳机制能够过滤非法消息和保证通信质量,在ROS2上层进行消息
变异容易被识别为非法,不能较好模拟实际的网路不稳定性。

ROS2系统支持使用不同的DDS协议中间件实现,如Fast-DDS、Cyclone-DDS等,从而灵活适配
不同的需求。利用这一点,本作品创新性地用魔改的DDS通信中间件替代原有基础DDS中间
件,如图\ref{pic:fmp}所示,直接在底层套接字通信实现中进行报文变异,从而更真实地
模拟底层网路的不稳定性。

\begin{figure}[h]
    \centering
    \includegraphics[width=15cm]{fuzz_msg_passing.png}
    \caption{本作品对不稳定网路的模拟机制}
    \label{pic:fmp}
\end{figure}

\subsubsection{多维度输入变异}

除了对上述ROS2内部通信消息进行变异,本作品同样也对传统程序输入进行变异,如输入
ROS2节点程序的命令、ROS2节点配置文件、用户交互数据等,见图\ref{pic:fmi}。通过多
个维度输入的变异,本作品可以高效地探索程序内部状态,更高效地尝试触发异常状态,检
测程序安全漏洞。

\begin{figure}[h]
    \centering
    \includegraphics[width=7cm]{fuzz_multi_input.png}
    \caption{本作品针对多维度输入进行测试样例生成}
    \label{pic:fmi}
\end{figure}

\subsubsection{基于延迟插入的并发漏洞检测}

如背景知识一节所述,ROS2系统有典型的并发特征,在节点间是基于消息通信的多进程并发
模型,而在节点内则是基于共享内存的多线程并发模型。基于消息通信的并发模型容易出现
死锁等阻塞漏洞,基于共享内存的并发模型则容易出现数据竞争等并发漏洞,这两点导致了
基于ROS2的程序存在较多并发漏洞,对ROS系统漏洞的统计工作$^{[3]}$也证明了这一点。

针对这一点,本作品尝试通过延迟插入技术来检测并发漏洞,工作$^{[1]}$已说明这种方法
具有漏洞探测的便捷性与高效性。通过举例来解释该技术:在正常程序中,当对象A被使用
时,它不应被释放或已被释放,对对象A的释放操作应阻塞等待所有对对象A的使用操作结束
再执行;本作品会尝试在某些边界情况中(如系统频繁申请与释放资源时)在对象A的使用
操作前插入延迟,如图\ref{pic:fi}右图,如果此时对A的释放操作没有正确阻塞等待对象A
的使用完毕,而直接执行,就会造成严重的UAF内存错误,因为此时使用的对象A已经被释
放。

\begin{figure}[h]
    \centering
    \includegraphics[width=15cm]{fuzz_delay_injection.png}
    \caption{延迟插入技术的探测漏洞原理}
    \label{pic:fdi}
\end{figure}

延迟插入技术可能拖慢整个程序的运行速度,所以具体插入位置需要谨慎选择。如前所
述,ROS2系统高度抽象化了各个子功能组件,将各个功能将回调函数绑定给执行器,执行器
负责监听事件并触发相应回调函数。统计$^{[3]}$显示这种线程模型是导致潜在数据竞争的
重要原因,由此,本作品基于LLVM框架对组件的回调函数前后插入检查点,如图
\ref{pic:fi}所示,对子线程可疑行为进行延迟插入,观察能否触发数据竞争漏洞。

\begin{figure}[h]
    \centering
    \includegraphics[width=7cm]{fuzz_instrumentation.png}
    \caption{延迟插入技术的实现}
    \label{pic:fi}
\end{figure}

\subsubsection{测试反馈}
模糊测试中,对“测试样例自动生成”的反馈指导质量直接影响了最终的测试效率,这种反馈
则取决于对被测程序的信息收集。传统被收集的信息包括调试信息、代码覆盖率等,这些信
息被认为反映了被测程序内部状态的变化。在此基础上,本系统创新性地利用我们魔改的
DDS中间件,对ROS2节点的流量进行分析,如图\ref{pic:ftm};同时利用插桩的检查点(如
图\ref{pic:fi})收集节点的功能组件回调函数执行信息,如执行时间、频率、传入的消息
内容等。这些额外信息更准确地反映了被测ROS2节点内部状态,指导我们更好地反馈生成新
测试样例,从而提高发掘漏洞的效率。

\begin{figure}[h]
    \centering
    \includegraphics[width=15cm]{fuzz_traffic_monitor.png}
    \caption{利用替换的DDS中间件进行流量分析}
    \label{pic:ftm}
\end{figure}


\subsubsection{测试加速}

复杂机器人系统通常资源消耗大,对本地网络污染严重,进程管理复杂,这导致了对其测试
的低效性。测试人员或自动测试程序难以实现对测试行为的并行化:本地网络中存在的多个
同名程序实例(如多个Navigation2导航程序)会被其他实例的通信消息所干扰,这是因为
ROS2系统会共享同一个通信守护进程;一个基于ROS2的项目大多是多进程多线程的,每个进
程有独立进程组,关闭程序时容易遗留孤立进程,持续干扰整个ROS2环境。

针对这一情况,本作品创新性地使用Docker对不同测试实例进行进程、网络的隔离;搭配使
用随机化ROS2命名空间,提供更轻量化的网络隔离,如图\ref{pic:fc}所示。通过这种方式,本作品可以充分利用电
脑性能进行并行化测试,变相大幅提升了自动测试效率,当然这也增加了本作品模糊测试程
序的开发难度。

\begin{figure}[h]
    \centering
    \includegraphics[width=15cm]{fuzz_container.png}
    \caption{测试的容器化与并行化}
    \label{pic:fc}
\end{figure}

%@author: yjw

\section{作品测试与成果分析}
\setParDis %设置段间距为 0
%@author: wkt
\subsection{测试环境介绍}
本作品在ROS2中测试了10个常见的机器人程序。根据最新版本的测试结果得到下列图表,表格1显示了这些用于测试的ROS程序的信息(源代码行数按CLOC计算{[}23{]})。为了执行机器人导航(用于移动和定位程序)和地图构建(用于移动和SLAM程序)任务,这些程序在机器人仿真框架Gazebo
11.5{[}24{]}中的虚拟机器人TurtleBot3
Waffle上执行。本测试使用的虚拟传感器包括激光雷达,里程计,2D相机和IMU(惯性测量单元)。实验运行在一台普通的x86-64台式机上,配有8个英特尔处理器和20GB物理内存。使用的操作系统为Ubuntu
20.04, ROS2版本为ROS2 Foxy。
\begin{table}[H]
	\small
	\caption{用于测试的ROS程序信息}
	\label{tb:ros_test}
	\centering
	\begin{tabular}{cccc}
		\hline  
		\textbf{类型} & \textbf{程序} & \textbf{描述} & \textbf{LOC} \\ 
		\hline  
		移动 & nav2\_bt\_navigator & ROS2导航中的BT导航模块 & 6.2K \\
		移动 & nav2\_planner & ROS2导航中的路径规划器模块 & 10.4K \\
		移动 & nav2\_recoveries & ROS2导航中的恢复模块 & 1.2K \\
		移动 & nav2\_controller & ROS2导航中的控制器模块 & 6.3K \\
		定位 & nav2\_amcl & ROS2导航中的定位模块 & 14.4K \\
		定位 & lama\_loc & 可选的定位和映射方法 & 9.5K \\
		定位 & ekf\_loc & 基于卡尔曼滤波的定位方法 & 0.8K \\
		SLAM & rtab-map & 实时RGB-D SLAM方法 & 26.1K \\
		SLAM & slam\_toolbox & 一套用于2D SLAM的工具和功能 & 15.1K \\
		SLAM & cartographer & 实时2D和3D SLAM方法 & 7.8K \\
		\hline
	\end{tabular} 
\end{table}

\subsection{测试成果与漏洞统计}
本测试运行一个通用的第三方分析工具ASan{[}20{]}和本作品来检测运行时的内存bug。每个程序都用相关的机器人任务进行24小时的测试。表格2显示了测试结果。

\begin{table}[H]
	\small
	\caption{机器人程序在ROS2模糊测试结果}
	\centering
	\begin{tabular}{ccccc}
		\hline
		\textbf{程序} & \multicolumn{2}{c}{\textbf{覆盖分支}} & \multicolumn{2}{c}{\textbf{漏洞检测}} \\
		\cline{2-5}
		& \textbf{模糊测试} & \textbf{测试样例} & \textbf{发现的漏洞} & \textbf{确认的漏洞} \\
		\hline
		nav2\_bt\_navigator & 24.3K & 17.2K & 3 & 2 \\
		nav2\_planner & 27.9K & 25.9K & 6 & 5 \\
		nav2\_recoveries & 15.8K & 16.4K & 2 & 2 \\
		nav2\_controller & 37.6K & 25.1K & 3 & 1 \\
		nav2\_amcl & 12.2K & 12.0K & 9 & 7 \\
		lama\_loc & 25.3K & 20.7K & 3 & 0 \\
		ekf\_loc & 16.2K & 5.2K & 1 & 0 \\
		rtab-map & 70.9K & 59.3K & 8 & 1 \\
		slam\_toolbox & 30.9K & 10.7K & 3 & 2 \\
		cartographer & 59.9K & 52.1K & 5 & 0 \\
		\textbf{总计} & \textbf{320.9K} & \textbf{244.5K} & \textbf{43} & \textbf{20} \\
		\hline
	\end{tabular}
\end{table}

\textbf{测试覆盖率。}为了理解本作品的测试覆盖改进,本测试还对每个ROS程序运行了24小时的官方测试用例,并收集了覆盖的代码分支的数量。得益于本作品基于ROS属性的模糊测试方法,与运行测试样例相比,本作品覆盖了测试程序中31\%的代码分支。值得一提的是,在运行官方测试用例时,本作品没有发现任何错误。

\textbf{漏洞检测。}本作品在测试程序中发现了43个真实的漏洞,没有误报。这些漏洞实际上可以使用本作品及其生成的测试用例来重现。本团队已经向相关ROS开发者报告了这些漏洞,其中20个漏洞已经得到了确认和修复,仍在等待对剩余漏洞的反馈。值得注意的是,有五个测试程序的名称包含``nav2'',它们是由ROS社区开发和维护的且广泛用于基于ROS的机器人。因此,这5个程序中已确认的17个bug受到了ROS开发者的高度关注。
\subsection{ROS漏洞实例分析}
本团队还根据发现的43个漏洞的类型进行了分类,并将结果总结在表3中。具体来说,有6个空指针解引用,5个释放后使用错误,3个缓冲区/堆栈溢出错误,11个无效指针访问和18个未捕获异常,如表格3所示。一旦这些错误被特定的输入触发,运行时故障和严重的安全问题就会在运行时发生。具体来说,空指针解引用和未捕获的异常可能导致程序崩溃,从而异常中止机器人任务;缓冲区/堆栈溢出错误、释放后使用错误和无效指针访问会导致机器人的未定义行为,并增加机器人受到恶意攻击的风险。下面将对不同种类的漏洞进行简要介绍其特点及危害。
\begin{table}[H]
	\small
	\caption{发现漏洞的种类}
	\centering
	\begin{tabular}{ccccccc}
		\hline
		\textbf{程序} & \textbf{空指针} & \textbf{释放后使用} & \textbf{溢出} & \textbf{无效指针} & \textbf{异常} & \textbf{总计} \\
		\hline
		nav2\_bt\_navigator & 1 & 0 & 0 & 0 & 2 & 3 \\
		nav2\_planner & 0 & 2 & 1 & 2 & 1 & 6 \\
		nav2\_recoveries & 0 & 2 & 0 & 0 & 0 & 2 \\
		nav2\_controller & 0 & 0 & 1 & 1 & 1 & 3 \\
		nav2\_amcl & 2 & 0 & 0 & 6 & 1 & 9 \\
		lama\_loc & 2 & 0 & 0 & 0 & 1 & 3 \\
		ekf\_loc & 0 & 0 & 0 & 0 & 1 & 1 \\
		rtab-map & 0 & 0 & 1 & 1 & 6 & 8 \\
		slam\_toolbox & 1 & 1 & 0 & 1 & 0 & 3 \\
		cartographer & 0 & 0 & 0 & 0 & 5 & 5 \\
		\textbf{总计} & \textbf{6} & \textbf{5} & \textbf{3} & \textbf{11} & \textbf{18} & \textbf{43} \\
		\hline
	\end{tabular}
\end{table}

本测试检测到的5个释放后使用漏洞是由数据竞争引起的。具体来说,一个内存对象在一个线程中被释放,但这个对象仍然在另一个线程中使用,没有同步。由于并发执行的不确定性,在正常执行中很难发现这些漏洞。一个真实含有释放后使用漏洞的代码如图表1所示

本测试检测到的这18个未捕获的异常中,有8个错误是由被测程序代码中的内部异常引起的,10个错误是由被测ROS程序使用的第三方软件库和ROS核心组件的API调用的外部异常(在c++中属于``运行时错误''类型)引起的。这个特性表明应该在ROS程序中小心地捕获和处理内部和外部异常。例如,在nav2\_bt\_navigator、nav2\_planner、nav2\_controller和nav2\_amcl中,没有捕捉到来自ROS
rclcpp组件的关于负时间间隔的四个外部异常,这可能导致程序崩溃。真实的代码案例如下图表2所示:

在ROS程序初始化过程中出现了11个bug。这些错误中有6个是由于初始化过程中对无效用户数据和不正确参数配置的错误处理引起的;还有5个漏洞是由初始化过程和消息处理之间缺少同步引起的。因此,在测试ROS程序时,应该特别注意初始化过程。

通过手工检查发现漏洞的回溯,本团队发现有7个漏洞位于被测试的ROS程序所使用的第三方库和ROS核心组件中。这些库和组件包括libopencv、fasttps、eigen、tf2和rclcpp。由于在实验中使用Asan仅仅检测被测试的ROS程序,而不是这些库或组件,因此无法找到这些漏洞的准确位置。为了解决这些问题,本作品尝试使用ASan来检测这些库和组件。但在尝试中发现,它们都不能支持ASan。即便如此,本团队在不使用ASan的情况下手动检查源代码并且成功在rclcpp中找到了一个漏洞(堆栈溢出)。目前rclcpp开发人员已经确认并修复了这个漏洞。

7个漏洞是由回调函数的并发性引起的。在ROS程序中,外部事件(如消息到达)随时会发生,并且每种事件均是由回调函数处理。由于事件可以在任何时间发生,它的回调函数可以与其他函数并发执行。因此,由于不正确的同步,回调函数中可能出现并发错误。图6(a)显示了nav2\_planner中的一个示例错误。回调函数footprint.callback可以与函数getFootprint并发执行。在footprint.callback函数中,指针footprint\_与msg在第106行一起被分配,因此由footprint\_所指向的内存会根据智能指针的功能被释放。同时,getFootprint中仍然使用这个智能指针来访问第76行中的footprint\_-\textgreater polygon,因此导致释放后使用的漏洞。{[}35{]}

11个漏洞是由错误处理问题引起的。具体来说,其中6个错误是由于对无效用户数据和错误的配置参数缺少或不正确的安全检查而引入的;另外5个错误是由于正确的安全检查后错误处理不正确而引入的。图6(b)显示了nav2\_controller中的一个示例错误。transformLaserScanToPointCloud函数可以在运行时抛出一个显式异常和一个隐式异常。在第294行只捕获显式异常,但未捕获``运行时错误''类型的隐式异常{[}36{]}。

2个漏洞是由堆栈内存问题引起的。其中一个漏洞是由在新线程中使用局部变量引起的,另一个漏洞是由过多的递归调用引起的。图6(c)显示了rtab-map中的一个示例错误。在CoreWrapper类的构造函数中,主线程的第141行定义了一个局部变量tfDelay,但是这个变量是在一个通过new
std::thread创建的新线程中的第590行被访问的。由于新线程无法访问主线程的堆栈,因此会出现堆栈溢出错误{[}37{]}。

\subsection{与同类技术对比}
本测试通过实验将本作品与两种最先进的机器人程序测试方法Ros2-fuzz{[}10{]}和ASTAA{[}38{]}进行了比较。ROS2
-fuzz是一种基于AFL{[}2{]}的自动化模糊测试方法,用于在ROS2中测试机器人程序。这种方法会对给定ROS节点中特定主题的消息进行变异。由于Ros2-fuzz是开源的,我们从源代码构建它。ASTAA是一种针对机器人程序鲁棒性测试的方法。它在ROS节点之间随机改变消息,并且还可以在运行时丢弃一些消息以模拟通信不稳定。由于ASTAA是闭源的,本团队通过修改本作品实现了一个类似ASTAA的工具,只允许传感器消息的数据突变和随机丢弃消息,而不使用程序反馈。

在实验中,我们在表1中选择了5个名称包含``nav2''的程序,并运行Ros2-fuzz、ASTAAlike工具和本作品对这些程序进行了24小时的机器人导航任务测试。表4显示了比较结果,其中包括覆盖的代码分支和发现的漏洞。
\begin{table}[H]
	\small
	\caption{不同测试方法的漏洞检测结果}
	\centering
	\begin{tabular}{ccccccc}
		\hline
		\multirow{2}{*}{\textbf{程序}} & \multicolumn{2}{c}{\textbf{Ros2-fuzz}} & \multicolumn{2}{c}{\textbf{ASTAA-like}} & \multicolumn{2}{c}{\textbf{本作品}} \\
		\cline{2-7}
		& \textbf{分支} & \textbf{发现漏洞数} & \textbf{分支} & \textbf{发现漏洞数} & \textbf{分支} & \textbf{发现漏洞数} \\
		\hline
		nav2\_bt\_navigator & 1.8K & 0 & 23.7K & 1 & 24.3K & 3 \\
		nav2\_planner & 1.6K & 0 & 26.4K & 2 & 27.9K & 6 \\
		nav2\_recoveries & 4.8K & 0 & 15.1K & 1 & 15.8K & 2 \\
		nav2\_controller & 3.1K & 0 & 35.2K & 3 & 37.6K & 3 \\
		nav2\_amcl & 0.7K & 0 & 11.9K & 2 & 12.2K & 9 \\
		\textbf{总计} & \textbf{12.0K} & \textbf{0} & \textbf{112.3K} & \textbf{9} & \textbf{117.8K} & \textbf{23} \\
		\hline
	\end{tabular}
\end{table}
本作品发现了Ros2-fuzz和ASTAA-like的工具发现的所有9个漏洞,并且它还发现了这些方法遗漏的14个漏洞,且具有更高的代码覆盖率。实际上,Ros2-fuzz和ASTAA仅生成关于ROS节点之间消息的测试用例,因此在模糊测试期间没有涉及处理不同用户数据和配置参数的大量代码。相比之下,本作品从多个维度(包括用户数据、配置参数和传感器消息)生成测试用例,因此本作品覆盖了Ros2-fuzz和ASTAA遗漏的更多代码。此外,本作品使用分布式分支覆盖更有效地指导多个ROS节点的测试用例生成,并使用三种常见模式执行时间突变(ASTAA只考虑其中一种模式,即消息丢弃),以更有效地覆盖有关时间特征的代码。由于这些原因,在实验中,本作品比Ros2-fuzz和类ASTAA-like的工具产生更好的结果。

通过分析被覆盖分支随着测试时间的增长,观察到随着时间的推移,这三种工具覆盖的新代码分支越来越少。这是因为许多代码分支已经被模糊测试期间由早期突变生成的测试用例所覆盖。即便如此,本团队观察到在后面的测试中,由于本项目的多维生成方法、分布式分支覆盖和时间突变策略,本作品覆盖了更多的新代码分支。

本作品也适用于在ROS1中测试机器人程序。因此,本团队使用本作品和ASan来测试了ROS1中三个常见的机器人程序,包括move\_base
{[}39{]}, nav1\_
amcl{[}40{]}和hector\_mapping{[}41{]}。对于机器人导航任务,本团队运行move\_base和nav1\_amcl,然后运行move\_base和hector\_mapping来完成地图建立任务。本团队对每个程序及其相关任务测试24小时。与第IV-A节类似,同时还运行每个程序的官方测试样例24小时,以验证本作品的测试覆盖改进。表5显示了测试结果。
\begin{table}[H]
	\small
	\caption{ROS1中机器人程序测试结果}
	\centering
	\begin{tabular}{cccccc}
		\hline
		\multirow{2}{*}{\textbf{程序}} & \multicolumn{2}{c}{\textbf{覆盖分支}} & \multicolumn{3}{c}{\textbf{发现的漏洞}} \\
		\cline{2-6}
		& \textbf{模糊测试} & \textbf{测试套件} & \textbf{无效指针} & \textbf{异常} & \textbf{总计} \\
		\hline
		move\_base & 36.4K & 31.9K & 0 & 2 & 2 \\
		navl\_amcl & 12.5K & 10.9K & 2 & 2 & 4 \\
		hector\_mapping & 13.1K & 11.5K & 0 & 0 & 0 \\
		\textbf{总计} & \textbf{62.0K} & \textbf{54.3K} & \textbf{2} & \textbf{4} & \textbf{6} \\
		\hline
	\end{tabular}
\end{table}

与运行测试样例相比,本作品在测试程序中覆盖了14\%的代码分支,这得益于我们基于ROS属性的模糊测试方法。值得注意的是,在运行官方测试样例时,我们没有发现任何错误。由于测试覆盖率的提高,本作品发现了6个真正的错误,没有误报,包括2个无效指针访问和4个未捕获的异常。本团队已经向相关ROS开发人员报告了这些报告,但尚未收到任何回应。实际上,这三个测试程序的github存储库已经很长时间没有更新了。

\section{创新型说明与前景分析}
\setParDis %设置段间距为 0
%@author: jfb
\subsection{创新性说明}
\setParDis %设置段间距为 0
\begin{enumerate}
  \item 创新地将模糊测试技术迁移
  \item 创新测试方法,高效性
\end{enumerate}

\section*{结论}% section*生成无标号章节
\addcontentsline{toc}{section}{结论} % 将无标号章节添加至目录
%@author: jfb
傻逼冯如杯

%注意: 文件大小不超过5M。%

\end{spacing}
%%%%%%%%%%%%%%%%%%%%引用部分
\newpage

% XSP 2023/3/16: bib支持不全,暂时改为手动
\section*{参考文献} % section*生成无标号章节题目
\addcontentsline{toc}{section}{参考文献} % 将无标号章节添加至目录
% 延迟插桩论文
[1]Stoica, B.A., Lu, S., Musuvathi, M., \& Nath, S. \textit{WAFFLE: Exposing Memory Ordering Bugs Efficiently with Active Delay Injection}. In EuroSys'23, May 2023. 

% ROZZ论文
[2]K. -T. Xie, J. -J. Bai, Y. -H. Zou and Y. -P. Wang. \textit{ROZZ: Property-based Fuzzing for Robotic Programs in ROS}. 2022 International Conference on Robotics and Automation (ICRA), Philadelphia, PA, USA, 2022, pp. 6786-6792.

% ROS Robust 221 bugs
[3]Timperley, C.S., van der Hoorn, G., Santos, A., Deshpande, H., \& Wasowski, A. \textit{ROBUST: 221 bugs in the Robot Operating System}. Empirical Software Engineering, 29(3), 57, 2024.

% in wkt.tex, use gazebo
[4]"Gazebo: a robot simulation framework." Available: http://gazebosim.org/.

% Asan
[5]Serebryany, K., Bruening, D., Potapenko, A., \& Vyukov, D. \textit{AddressSanitizer: A Fast Address Sanity Checker}. In USENIX ATC 2012, 2012.

% 机器人相关fuzz,用于同类技术对比, 当然也包括[2]
[6] Delgado, R., Campusano, M., \& Bergel, A. \textit{Fuzz testing in behavior-based robotics}. In Proceedings of the 2021 International Conference on Robotics and Automation (ICRA), 9375–9381, 2021.

[7] Woodlief, T., Elbaum, S., \& Sullivan, K. \textit{Fuzzing mobile robot environments for fast automated crash detection}. In Proceedings of the 2021 International Conference on Robotics and Automation (ICRA), 5417–5423, 2021.

[8] Hutchison, C., Zizyte, M., Lanigan, P.E., Guttendorf, D., Wagner, M., Le Goues, C., \& Koopman, P. \textit{Robustness testing of autonomy software}. In Proceedings of the 40th International Conference on Software Engineering: Software Engineering in Practice Track (ICSE-SEIP), 276–285, 2018.

[9] Kim, S. \& Kim, T. \textit{RoboFuzz: fuzzing robotic systems over robot operating system (ROS) for finding correctness bugs}. In Proceedings of the 30th International Symposium on the Foundations of Software Engineering (FSE), 447–458, 2022.

% ros2-fuzz
[10] "Ros2-fuzz: automatic fuzzing for ROS2." Available: https://github.com/rosin-project/ros2 fuzz.

% AFL
[11] Zalewski, M. \textit{American Fuzzy Lop: A Security-Oriented Fuzzing Tool}. In Proceedings of the Black Hat USA, 2015.


% \begingroup
% \setstretch{2.0}    %行距2
% \setlength{\bibsep}{0pt}    %段前段后0
% \begin{adjustwidth}{0.42cm}{0.42cm} %左右缩进0.42cm
% \bibliography{references}
% \end{adjustwidth}
% \endgroup

\end{document}
