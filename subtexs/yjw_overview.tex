% author: yjw
本作品给出针对ROS2系统的自动模糊测试程序,其核心框架如图\ref{pic:off}所示,旨在
提高对ROS2系统相关的自动驾驶项目、机器人项目的漏洞测试的便捷性和高效性,为开发者
提供一个有效测试工具。开发者首先将源码搭配本作品的编译器工具编译为可执行程序,此
过程中本作品将自动使用LLVM架构对源码进行插桩修改,并链接相应检测分析程序(如Address
Sanitizer等第三方监控工具);接着,本作品核心模糊测试驱动器会启动特殊ROS2执行环
境执行开发者的ROS2程序,并将开发者提供的初始测试样例进行变异修改,生成一个测试用
例输入进程序;本作品的监控器(Node Monitor)会对被测程序的代码覆盖率、程序异常、
通信流量、服务状态等信息进行收集分析;分析结果被作为该测试用例种群适应度输入到遗
传算法中,遗传算法会对种群(测试用例池)进行进化和淘汰,选择出更有可能触发程序边
界的测试用例,并指导对测试用例种群进行进一步变异。

\begin{figure}[H]
    \centering
    \includegraphics[width=14cm]{our_fuzz_framework.png}
    \caption{本项目实现的Fuzz框架}
    \label{pic:off}
\end{figure}

通过上述步骤,本作品会不断生成被测ROS2节点程序的高质量测试用例,覆盖更多节点程序
的内部状态分支,以更高效率探索程序潜在漏洞。本作品的创新之处主要有几点:针对ROS2
系统程序特点,不局限于传统测试输入,探索多维度的程序输入;以更高效的办法对ROS2节
点程序进行监控,综合白盒插桩检测与黑盒流量分析等技术;支持检测多种漏洞类型,如内
存安全漏洞与并发漏洞等。

% 终极版本是一个魔改的ROS环境,包括魔改DDS,魔改RCLCPP链接库,Tracing守护进程(流量分析),一个Node可以不运行在其项目整个系统中,如测试Navigation2中的Nav2 AMCL可以不运行整个系统,魔改ROS环境会自动接管其所有接口,进行输入变异
