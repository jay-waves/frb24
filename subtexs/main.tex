\documentclass[zihao=-4]{ctexart}
\usepackage[normalem]{ulem}
\useunder{\uline}{\ul}{}
%********************导言区宏包引入********************
\usepackage{xeCJK}
\usepackage{amssymb}
\usepackage{amsmath}
\usepackage{listings} %代码
\usepackage{graphicx}
\usepackage{multicol} %回车换段
\usepackage{xcolor}
\usepackage{geometry} %页面设置
\usepackage{fontspec}
\usepackage{setspace}
\usepackage{times}
\usepackage{fancyhdr} %页眉页脚
\pagestyle{fancy}
\usepackage{float} %表格位置
\usepackage{titlesec} %设置
\usepackage{titletoc}
\usepackage{ctex}
\usepackage{gbt7714}    %控制参考文献格式为国标
\usepackage{multirow}
\usepackage{booktabs}   %表格相关
\usepackage{setspace}   %设置行距
\usepackage{caption} %caption
\usepackage{subcaption} %子图的caption
\usepackage{changepage} %左右缩进


\graphicspath{ {pictures/} }
\let\algorithm\relax
\let\endalgorithm\relax
\usepackage[ruled,vlined]{algorithm2e}%[ruled,vlined]{
\usepackage{algpseudocode}
\renewcommand{\algorithmicrequire}{\textbf{Input:}} 
\renewcommand{\algorithmicensure}{\textbf{Output:}}
%\renewcommand\thepage{\zihao{-5} ~\arabic{page}~}%页码字号

%定义两个arg
\DeclareMathOperator*{\argmax}{arg\,max}
\DeclareMathOperator*{\argmin}{arg\,min}
\DeclareCaptionLabelSeparator{mysep}{\space\space}  %自定义caption格式
\captionsetup[figure]{font={small}, labelfont=bf, labelsep=mysep, textfont=bf}   %图片caption格式
\captionsetup[table]{font={small}, labelfont=bf, labelsep=mysep, textfont={bf}}   %表格caption格式
\bibliographystyle{gbt7714-numerical} %修改了title斜体内容

%********************导言区宏包引入********************
%********************第三方字体引入********************
%\setCJKmainfont[Path=fonts/,BoldFont=simhei.ttf,ItalicFont=simkai.ttf,SlantedFont=simfang.ttf]{simsun.ttc}
%中文字体涵盖黑体、宋体、楷体、仿宋
\setmainfont[Path=fonts/, 
BoldFont = times-new-roman-bold.ttf,
ItalicFont = times-new-roman-italic.ttf,
BoldItalicFont = times-new-roman-bold-italic.ttf
]{times-new-roman.ttf}
\setmonofont[Path=fonts/]{Courier New.ttf}
\setCJKfamilyfont{hwzs}[Path=fonts/]{STKzhongsong.ttf}%使用STZhogsong华文中宋字体
\newcommand{\zhongsong}{\CJKfamily{hwzs}}
\setCJKfamilyfont{hwxw}[Path=fonts/]{STKxinwei.ttf} % XSP 2023/3/3:
\newcommand{\xinwei}{\CJKfamily{hwxw}}              %  使用STZxinwei华文新魏字体.

%********************第三方字体引入********************


%********************中文字号设置********************
%\newcommand{\chuhao}{\fontsize{42pt}{\baselineskip}\selectfont}
\newcommand{\chuhao}{\fontsize{42pt}{0}}
\newcommand{\xiaochu}{\fontsize{36pt}{0}}
\newcommand{\yihao}{\fontsize{28pt}{0}}
\newcommand{\erhao}{\fontsize{21pt}{0}}
\newcommand{\xiaoer}{\fontsize{18pt}{0}}
\newcommand{\sanhao}{\fontsize{16pt}{0}}
\newcommand{\sihao}{\fontsize{14pt}{0}}
\newcommand{\xiaosi}{\fontsize{12pt}{0}}
\newcommand{\wuhao}{\fontsize{10.5pt}{0}}
\newcommand{\xiaowu}{\fontsize{9pt}{0}}
\newcommand{\liuhao}{\fontsize{8pt}{0}}
\newcommand{\qihao}{\fontsize{5.25pt}{0}}
%********************中文字号设置********************


%********************页边距设置********************
\geometry{left=3cm,right=2cm,top=2.5cm,bottom=2.5cm}
\geometry{a4paper} % xsp 2023/3/7: 调整纸张大小为A4
%********************页边距设置********************

%********************段间距设置********************
\newcommand{\setParDis}{\setlength {\parskip} {0pt} }
%请在每部分使用这个
%********************段间距设置********************

%********************

\begin{document}
%********************页眉页脚设置********************
\lhead{}%设置左页眉为空
\rhead{}%设置左页眉为空
\setlength{\headwidth}{\textwidth}% 2023/3/3 XSP: 页眉长度适应文本
%********************页眉页脚设置********************


%********************标题格式设置********************

%\setcounter{secnumdepth}{0}%该命令取消了章标题前数字label

\CTEXsetup[name={,、},number={\chinese{section}}]{section}
\CTEXsetup[name={(,)},number={\chinese{subsection}}]{subsection}
\CTEXsetup[name={,.},number={\arabic{subsubsection}}]{subsubsection}% 不加会导致目录格式错误
% 设置subsubsection等格式
% \titleformat{\section}[block]{\sanhao\bfseries\centering}{\chinese{section}、}{0pt}{}[]
% \titleformat{\subsection}[block]{\sihao\bfseries}{(\chinese{subsection})}{0pt}{}[]
% \titleformat{\subsubsection}[block]{\xiaosi\bfseries}{\arabic{subsubsection}、}{0pt}{}[]
\titleformat{\section}[block]{\sanhao\heiti\centering}{\chinese{section}、}{0pt}{}[]    % XSP 2023/3/3:
\titleformat{\subsection}[block]{\sihao\heiti}{(\chinese{subsection})}{0pt}{}[]       %   将正文标题字体由加粗
\titleformat{\subsubsection}[block]{\xiaosi\heiti}{\arabic{subsubsection}.}{0pt}{}[]   % 修改为黑体。
\titlespacing{\section}{0pt}{25pt}{12pt}
\titlespacing{\subsection}{0pt}{7pt}{7pt}
\titlespacing{\subsubsection}{0pt}{5pt}{4pt}

\titlecontents{section}[1.6em]{\addvspace{2pt}\filright}
{\contentspush{\thecontentslabel\hspace{0.8em}}}
{}{\titlerule*[8pt]{.}\contentspage}

\titlecontents{subsection}[3.2em]{\addvspace{2pt}\filright}
{\contentspush{\thecontentslabel\hspace{0.8em}}}
{}{\titlerule*[8pt]{.}\contentspage}

\titlecontents{subsubsection}[6.4em]{\addvspace{2pt}\filright}
{\contentspush{\thecontentslabel\hspace{0.8em}}}
{}{\titlerule*[8pt]{.}\contentspage}
%********************标题格式设置********************

%\setcounter{section}{-3}  %标题计数器
%\stepcounter{section}

%*******************行间距段前段后*******************
\linespread{1.8}
%行间距为实际行间距乘以1.2,如此处实际为1.5倍行距
\setlength{\parskip}{0.5\baselineskip}
%*******************行间距段前段后*******************



%********************封面部分********************
%
%     论文题目:应准确、鲜明、简洁,能概括整个论文中最主要和最重要的内容。
% 题目不超过20个中文字,若语意未尽,可用副标题补充说明。副标题应处于从属
% 地位,一般可在题目的下一行用破折号“——”引出。论文题目应避免使用不常用缩
% 略词、首字母缩写字、字符、代号和公式等。
%
\def\Fengru{第三十四届“冯如杯”竞赛主赛道}
\leftline{\includegraphics[scale=1]{pictures/xiaohui.png}} % XSP 2023/3/3: 取消校徽段首缩进
%格式控制部分
% \par \  
% \par \
% \par \
\vspace{32pt}
\begin{center}
\includegraphics[height=2.25cm, width=12.78cm, scale=1]{pictures/xiaoming.png}
\end{center}
%格式控制部分
\vspace{12pt}

\begin{spacing}{3}
    % \erhao
    \begin{center}
      {
        \fontsize{22pt}{3}\selectfont
        \zhongsong{\Fengru 项目论文Latex模板} %黑体这样调用,其余字体同理
      } 
        % \zhongsong{“冯如杯”竞赛主赛道项目是什么}
    \end{center}
    \rightline{\xinwei\sanhao{——基于 Latex 的论文模板}} % XSP 2023/3/3: 副标题二号华文新魏居右
\end{spacing}
%格式控制部分
% \par \ 
% \par \
\par \ 
\par \
\par \ 
\par \
% \begin{center}
%     \sihao
%     \textbf{学院:计算机学院}
%     \par \ 
%     \textbf{本模板原作者:Someday}
% \end{center}

%格式控制部分
\par \ 
\begin{center}
\sanhao
\centerline{\heiti{}}%封面年月去掉
\end{center}

\pagenumbering{gobble} %封面无页码
%\thispagestyle{empty}


\renewcommand{\headrulewidth}{0pt}%没有页眉装饰线
\clearpage
\pagenumbering{roman} %摘要目录页小写罗马

\xiaosi
\section*{摘要}
\begin{spacing}{1.5}
  \setParDis %设置段间距为 0
  本Latex模板是北京航空航天大学大学\Fengru 论文模板, 由北京航空航天大学校团委
基于GitHub用户\textbf{\textit{Somedaywilldo}}与\textbf{\textit{cpfy}}的成果迭代
开发而来。在此由衷感谢所有开发者对本模板的贡献与对“冯如杯”竞赛的大力支持。

摘要内容包括:“摘要”字样,摘要正文,关键词。在摘要的最下方另起一行,用显著的字符注明文本的关键词。

摘要是论文内容的简短陈述,应体现论文工作的核心思想。摘要一般约500字。摘要内容应涉及本项科研工作的目的和意义、研究思想和方法、研究成果和结论。

关键词是为用户查找文献,从文中选取出来用来揭示全文主题内容的一组词语或术语,应尽量采用词表中的规范词(参照相应的技术术语标准)。关键词一般为3到8个,按词条的外延层次排列。关键词之间用逗号分开,最后一个关键词后不打标点符号。

\end{spacing}
    
\textbf{关键词:}关键词1,关键词2,关键词3,关键词4,关键词5

\newpage
\section*{\textbf{Abstract}} % XSP 2023/3/8: Abstract 加粗
\begin{spacing}{1.5}
\begin{adjustwidth}{0.42cm}{0.42cm}
  \setParDis %设置段间距为 0

\qquad This Latex template for the 33rd Fengru Cup Competition of 
Beihang University, is developed by Communist Youth League Committee of BUAA 
iteratively based on the contribution of GitHub 
users \textbf{\textit{Somedaywilldo}} and \textbf{\textit{cpfy}}. 
Here, we would like to thank all the developers for their 
contributions to this template and for their support of the Fengru Cup Competition.

The abstract includes: the word "Abstract", the body of the abstract, and the keywords. On a separate line at the bottom of the abstract, indicate the key words of the text in prominent characters.

The abstract is a short statement of the content of the paper and should reflect the core ideas of the paper work. The abstract is usually about 500 words. The abstract should cover the purpose and significance of this scientific work, research ideas and methods, research results and conclusions.

Keywords are a set of words or terms selected from the text to reveal the subject content of the whole text for the user to find the literature, and the standardized words in the word list (refer to the corresponding technical terminology standards) should be used as much as possible. The keywords are usually 3 to 8, arranged according to the level of extensibility of the words. The keywords are separated by commas, and no punctuation marks are used after the last keyword.

\textbf{Keywords: Keywords 1, Keywords 2, Keywords 3, Keywords 5, Keywords 6}
\end{adjustwidth}
\end{spacing}



%********************摘要部分********************


%********************目录部分********************
\clearpage
\tableofcontents
\clearpage
%********************目录部分********************



\renewcommand{\headrulewidth}{0.4pt} %恢复页眉装饰线

%********************正文页眉部分********************
%\lhead{} 
\chead{\xiaowu 北京航空航天大学\Fengru 参赛作品} %设置居中页眉
%********************正文页眉部分********************

\pagenumbering{arabic} %正文页码从1开始,用阿拉伯数字
\setcounter{page}{1} 

%%%%%%%%%%%%%%%%%%%%%% 一 作品概述
\section{作品概述}
\setParDis %设置段间距为 0
\begin{spacing}{1.5} % 行距1.5


\subsection{背景介绍}
%@author: lhy
\subsubsection{ROS系统重要性}
人类生活与机器人密不可分。传统上,机器人在农业和制造业中被广泛用于任务自动化。最近的发展让机器人更加贴近每个人的日常生活。例如,亚马逊和谷歌正在部署无人配送系统[22,23],使用无人驾驶飞行器,机器人吸尘器的市场规模年增长率达到23\%[24],以及在2012年到2018年间,使用机器人进行手术的比例从2\%增加到15\%[25],显示出机器人产业为适应人类需求而快速增长的态势。同时,现代机器人,如自动驾驶车辆,正在变得更加复杂,要求集成复杂的子系统,例如感知、感觉、规划和执行。

为了应对开发复杂机器人系统的更高需求,机器人操作系统(ROS)[1]正在获得越来越多的关注。ROS是一个开源的中间件套件,用于机器人开发,它具有分布式机器人进程的消息传递机制、硬件抽象、广泛的开发工具(例如,模拟器)和机器人库(例如,路径规划算法)。使用ROS,开发者可以加速开发过程,无需重新发明轮子,而是可以专注于他们机器人的核心功能。凭借其多语言和多平台支持的理念,ROS正在成为机器人编程中的实际标准;它已被广泛采用于工业[26,27]、军事[28,29]、研究机构以及个人,预计到2024年将为55\%的商业机器人提供动力[30]。近些年ROS系统相关论文数量与Wiki用户数量如图1所示:

\begin{figure}[H]
  \centering
  \includegraphics[width=0.8\textwidth]{./pictures/ROS.png}
  \caption{ROS系统相关论文数量与Wiki用户数量}
\end{figure}

机器人操作系统(ROS)是一个开源平台,它为用C++和Python等不同语言实现机器人软件提供了许多实用的工具和库。超过740家商业公司正在使用ROS[2]来开发数百种实际应用的机器人[3],这表明了其在机器人开发中的流行度。然而,开发可靠的ROS程序具有挑战性,因为机器人可以在复杂的物理环境中工作,并可能遇到各种异常情况(如无效的配置参数和异常的传感器信息)。如果ROS程序不够可靠,它们的错误可能在与人类和其他机器人交互时导致危险事故。

ROS本身的缺陷或者开发者在应用中常用的ROS使用方式可能会影响依赖于ROS的广泛机器人系统,并对许多用户的安全和保障造成严重破坏。例如,ROS版本1没有认证的概念,允许网络上的任何实体完全访问任何机器人系统,窃听内部消息,甚至在知道IP地址和端口的情况下劫持执行。实际上,在互联网上扫描默认的ROS端口两个月后[31],发现部署在28个国家的超过100个ROS系统,这些系统完全暴露于此类攻击之下。ROS的最新版本(即ROS 2)在设计和实现时考虑了这些问题,并邀请了更广泛的公众对安全性进行审查。不幸的是,现有的工作和解决方案要么专注于关于认证和授权方法的网络安全方面[32,33,34,35],要么专注于回归测试[36],使得ROS社区在寻找影响机器人系统的健壮性和正确性的未知错误方面缺乏一种系统的测试方法。
\subsubsection{自动驾驶兴起与ROS}
自动驾驶技术的兴起标志着交通运输领域的一次重大革命,其不仅预示着交通安全、效率和环境影响的根本改进,而且还预示着个人出行和货物运输方式的深刻变革,而这一过程在很大程度上得益于机器人操作系统(Robot Operating System,ROS)的发展和应用。近年来,由于计算能力的显著提升、大数据技术的发展、人工智能算法的进步以及传感器技术的优化,自动驾驶汽车从理论研究和小规模试验逐步走向了公路测试和商业化应用的初步阶段。

根据国际汽车工程师学会(SAE)的定义[1],自动驾驶分为六个级别,从0级(无自动化)到5级(完全自动化)。目前,多数公开测试和部分商业化应用的自动驾驶汽车处于3级(有条件自动化)到4级(高度自动化)。尽管完全自动化(5级)的汽车尚未广泛部署,但多个技术开发者和制造商已经在进行相关的研发工作,并在不同国家和地区开展了路试。

随着自动驾驶技术的进步,对处理大量传感器数据、实现复杂决策逻辑和执行精确控制的需求不断增加,ROS的可扩展性和灵活性在此过程中显示出其不可或缺的价值。例如,ROS的消息传递系统支持多种编程语言,允许异构系统高效集成。此外,其庞大的开源社区贡献了大量的软件包,覆盖从3D视觉到路径规划等各种功能,极大地丰富了自动驾驶汽车的研发资源库。

经济学人智库的一份报告预测[2],到2035年,全球自动驾驶汽车的数量将达到近7500万辆。而麦肯锡公司的一项研究则估计[3],自动驾驶技术的全面部署将使交通事故造成的死亡人数减少90\%,同时还将大幅度提升道路运输效率和减少碳排放。图2展示了PRECEDENCE RESEARCH机构预测的未来自动驾驶市场规模:

\begin{figure}[H]
  \centering
  \includegraphics[width=0.8\textwidth]{./pictures/market_size.png}
  \caption{自动驾驶市场规模}
  \label{fig:my_label}
\end{figure}

尽管自动驾驶技术的前景被广泛看好,但其商业化应用和普及还面临多重挑战,包括技术完善度、法律法规、道德伦理、消费者接受度以及基础设施配套等。例如,自动驾驶汽车在复杂的城市交通环境中的表现,以及在极端天气条件下的可靠性,仍然是技术研发中的重要挑战。

未来,随着相关技术的进一步成熟和相关政策的完善,自动驾驶汽车有望在更广泛的领域得到应用,从而实现对交通系统的根本性改造和优化。这将不仅影响交通运输本身,还将对城市规划、能源消费、环境保护以及人们的生活方式产生深远影响。

\subsubsection{自动驾驶产生漏洞的危害}
尽管自动驾驶汽车(AVs)承诺通过减少人为错误来增加道路安全,它们固有的技术漏洞却可能导致严重的安全后果。据国家公路交通安全管理局(NHTSA)估计,94\%的严重交通事故是由人为因素引起的,而自动驾驶技术的开发者们宣称,通过消除这一因素,可以极大地提高道路安全性[4]。然而,这种安全性的提升假设基于自动驾驶系统能够无缺陷地执行其预定任务,这在实践中往往难以实现。

自动驾驶汽车的技术漏洞主要分为两类:软件漏洞和硬件故障。软件漏洞包括但不限于算法错误、安全漏洞以及对异常情况的处理不足。硬件故障可能涉及传感器故障、执行机构的故障或其他关键组件的损坏。这些技术漏洞不仅可能导致单一车辆的操作失败,还可能对整个交通系统产生连锁反应,特别是在高度自动化和互联的交通环境中。

例如,2018年3月,在美国亚利桑那州发生了一起致命的自动驾驶汽车事故,一名行人在穿越街道时被一辆自动模式下的Uber汽车撞击,导致死亡。初步调查显示,该事故部分原因是系统无法正确识别并响应行人[5]。这一事件凸显了即使在自动驾驶技术取得显著进步的情况下,技术漏洞仍然可能导致灾难性后果。

此外,安全研究人员已经证明,自动驾驶汽车系统的安全漏洞可以被黑客利用,进而控制车辆的行驶方向或速度,造成严重安全威胁[6]。例如,通过篡改车辆的环境感知系统接收到的数据,攻击者可以使汽车忽略停车标志或其他关键交通信号,导致交通事故。

综上所述,尽管自动驾驶技术的发展被寄予厚望,能够在根本上改变我们的交通系统,降低事故率,提高交通效率,但技术漏洞的存在表明,要实现这些目标,还需要克服重大的技术和安全挑战。这要求汽车制造商、技术开发者、立法者和监管机构合作,不断提高自动驾驶系统的安全性和可靠性,同时为可能出现的技术漏洞和安全威胁制定全面的应对策略。

\subsection{研究现状}
\subsubsection{ROS常见测试方案}
%@author: yjw
普通单元测试, 代码分析工具
\subsubsection{模糊测试及其应用}
%@author: jfb
强调:模糊测试目前很火,但是没有用到ros上

\subsection{作品概述}
%@author: yjw


%%%%%%%%%%%%%%%%%%%%%% 二 作品设计与实现
\section{作品设计与实现}
\subsection{系统需求分析}
%@author: wkt
\begin{enumerate}
  \item 高覆盖率
  \item 高效性快速收敛
  \item 可拓展性
  \item 准确性高
\end{enumerate}

\subsection{技术背景与预备知识}
%@author: jfb
\subsubsection{模糊测试框架}
\subsubsection{漏洞种类及危害}
%@author: wkt
\subsubsection{ASan}
\subsubsection{基于覆盖率的遗传算法}
\subsubsection{ROS机制}
%@author: yjw

\subsection{关键设计}
%@author: yjw

\section{作品测试与成果分析}
%@author: wkt
\subsection{测试环境介绍}
\subsection{测试成果与漏洞统计}
\subsection{ROS漏洞实例分析}
\subsection{实例分析}
\subsection{与同类技术对比}

\section{创新型说明与前景分析}
%@author: jfb
\subsection{创新性说明}
\begin{enumerate}
  \item 创新地将模糊测试技术迁移
  \item 创新测试方法,高效性
\end{enumerate}

\subsection{前景分析}
\begin{enumerate}
  \item 创新维度
  \item 团队维度
  \item 商业维度
  \item 就业维度
  \item 社会服务
\end{enumerate}

\section*{结论}% section*生成无标号章节
\addcontentsline{toc}{section}{结论} % 将无标号章节添加至目录
%@author: jfb
傻逼冯如杯

%注意: 文件大小不超过5M。%

\end{spacing}
%%%%%%%%%%%%%%%%%%%%引用部分
\newpage

% XSP 2023/3/16: bib支持不全,暂时改为手动
\section*{参考文献} % section*生成无标号章节题目
\addcontentsline{toc}{section}{参考文献} % 将无标号章节添加至目录
% 著作: [序号]作者.书名[标识码].出版地:出版社,出版年.
[1]张志建.严复思想研究[M].桂林:广西师范大学出版社,1989. 

% 译著: [序号]国名或地区(用圆括号)原作者.书名[标识码].译者.出版地:出版社,出版年.
[2](英)霭理士.性心理学[M].潘光旦译.北京:商务印书馆,1997.

% 古典文献 文史古籍类引文后加序号,再加圆括号,内加注书名、篇名

% 论文集: [序号]编者.书名[标识码].出版地:出版社,出版年.
[3]伍蠡甫.西方论文选(下册)[C].上海:上海译文出版社,1979.

% 期刊文章: [序号]作者.篇名[标识码].刊名,年,(期).
[4]叶朗.《红楼梦》的意蕴[J].北京大学学报(哲学社会科学版),1989,(2)

% 报纸文章: [序号]作者.篇名[标识码].报纸名,出版日期(版次)
[5]谢希德.创造学习的新思路[N].人民日报,1998-12-25(10)

% 外文文献: 要求外文文献所表达的信息和中文文献一样多,但文献类型标识码可以不标出。
[6]Mansfeld, R.S. \& Busse. \textit{T.V. The Psychology of creativity and discovery}, Chinago:
NelsonHall, 1981




% \begingroup
% \setstretch{2.0}    %行距2
% \setlength{\bibsep}{0pt}    %段前段后0
% \begin{adjustwidth}{0.42cm}{0.42cm} %左右缩进0.42cm
% \bibliography{references}
% \end{adjustwidth}
% \endgroup

\end{document}
