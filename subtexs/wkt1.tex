\begin{enumerate}
	\item 高覆盖率 \\
	ROS系统的架构复杂,包含多个节点和话题,而本作品能够系统性地探索这些节点和话题
	之间的交互,从而实现全面覆盖。通过生成各种可能的输入和交互序列,本作品能够有效
	地发现潜在的漏洞和异常情况,为系统的安全性和稳定性提供保障。例如,它可以模拟不
	同的传感器输入和控制命令,以验证ROS系统对于各种情境的响应是否符合预期,从而增
	强系统的健壮性和可靠性。通过分布式分支覆盖,它能够捕获不同节点之间的代码执行路
	径,从而更全面地测试机器人程序。本作品使用多维生成方法生成ROS程序的测试用例,
	包括用户数据、配置参数和传感器消息。这种方法有助于覆盖不同维度的输入空间,提高
	测试的全面性。
	\item 高效性快速收敛 \\
	通过自动化生成和执行测试用例,本作品大幅提高了测试效率。快速生成大量测试用例,
	并采用智能化的测试执行策略,使得工具能够在短时间内快速收敛到潜在问题,帮助开发
	人员及时发现和修复bug,从而提高开发周期效率。举例来说,本作品可以根据程序执行
	路径的优先级和覆盖情况,动态调整测试用例的生成和执行顺序,优先测试具有潜在问题
	的部分,从而更快地发现关键问题。
	\item 可拓展性 \\
	由于ROS系统的多样性和复杂性,模糊测试工具必须具备良好的可拓展性,以应对不同类
	型和规模的ROS应用。这种工具通常设计成模块化的结构,能够轻松集成新的测试策略和
	技术,满足不断变化的ROS系统和应用场景的需求,使得测试工作更加灵活和适应性更
	强。例如,开发人员可以根据具体需求扩展工具的测试生成器、执行器或分析器,以适应
	新的ROS功能或更复杂的系统结构,从而提高测试的适用性和覆盖范围。
	\item 准确性高 \\
	在生成测试用例时,工具会考虑ROS系统的特性和约束,例如消息格式、节点通信方式
	等,以确保生成的测试用例有效且具有代表性。通过使用各种静态和动态分析技术,工具
	能够准确检测潜在问题和异常情况,并提供详尽的测试反馈和报告,为开发人员提供准确
	的问题定位和解决方案,从而提高系统的质量和可靠性。例如,工具可以监视ROS节点之
	间的消息传递,分析消息格式和内容,以及节点的响应时间,从而发现潜在的性能瓶颈或
	通信异常,帮助开发人员改进系统的设计和实现。本作品使用时态变异策略生成带有时间
	信息的测试用例。这有助于模拟实际机器人系统中的时间相关行为,提高测试的准确性。
\end{enumerate}