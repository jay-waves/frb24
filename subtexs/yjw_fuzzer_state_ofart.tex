%TODO: 审核,文献,表格格式
近期有几种$^{[2, 6, 7, 8, 9, 10]}$应用于ROS2程序的新型测试方法,它们不同于由开发者手动构造测试样例的单元测试,而尝试自动生成ROS程序输入,并借助清洁工具(Sanitizer)自动监控程序内存或语义错误。它们大幅提高了针对机器人系统的测试效率,表\ref{tab:fuzzers}将这些方法进行了比较(包括本作品),可以看出这些方法在测试ROS程序时仍存在三个主要限制:

\textbf{限制一:}测试用例生成效率低。大部分测试方法从单一维度(即用户命令或传感器信息)生成输入,并且生成的测试样例随机性过强,无法识别出可能高效探索程序边界的变异维度,导致许多生成的输入无助于增加测试覆盖率。

\textbf{限制二:}程序反馈效果不佳。SMACH-Fuzz和ASTAA随机生成输入,而没有反馈和指导;Ros2-fuzz和ROZZ参考AFL工具$^{[11]}$使用代码覆盖率作为程序反馈,但代码覆盖率对多进程运行情况不敏感,且忽略了不同执行路径;Phys-Fuzz使用场景危险程度评分来指导进一步生成场景,对于检测ROS2自动导航程序的避障能力效果较好,但也仅限制在了此场景下;RoboFuzz使用语义反馈来量化机器人执行上下文的正确性,但是这种特殊语义的设计编写需要很强专业知识。

\textbf{限制三:}自动化水平低。大部分方法都需要大量领域特定知识和手动努力来配置输入生成规则(ASTAA和RoboFuzz),编写有关机器人行为的规格(SMACH-Fuzz和RoboFuzz),检查执行日志(SMACH-Fuzz和Phy-Fuzz)等,这导致测试效率和便捷程度对比传统单元测试并没有很大提高。

\begin{table}[H]
\small
\centering
\caption{最新针对ROS系统的测试技术对比}
\begin{tabular}{lccccc}
\hline
\textbf{技术名} & \textbf{输入维度}  & \textbf{反馈}& \textbf{自动化程度} & \textbf{漏洞类型} \\ \hline
SMACH-Fuzz $^{[6]}$ & 单一 & 无 & 差 & 行为错误 \\ 
Phys-Fuzz $^{[7]}$ & 单一  & 环境危险程度  & 差& 碰撞损坏 \\ 
Ros2-fuzz $^{[10]}$ & 单一  & 代码覆盖率   & 较好 & 内存漏洞 \\ 
ASTAA $^{[8]}$ & 单一 & 无 & 差 & 行为错误 \\ 
RoboFuzz $^{[9]}$ & 单一  & 语义反馈 & 较差 & 行为错误 \\ 
ROZZ $^{[2]}$ & 多维度 & 代码覆盖率 & 较好 & 内存漏洞 \\ 
本作品 & 多维度  & 代码覆盖率+流量分析 & 较好& 内存与并发漏洞 \\ \hline
\end{tabular}
\label{tab:fuzzers}
\end{table}
