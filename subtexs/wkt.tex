\subsection{测试环境介绍}
本作品在ROS2中测试了10个常见的机器人程序。根据最新版本的测试结果得到下列图表,表格1显示了这些用于测试的ROS程序的信息(源代码行数按CLOC计算{[}23{]})。为了执行机器人导航(用于移动和定位程序)和地图构建(用于移动和SLAM程序)任务,这些程序在机器人仿真框架Gazebo
11.5{[}24{]}中的虚拟机器人TurtleBot3
Waffle上执行。本测试使用的虚拟传感器包括激光雷达,里程计,2D相机和IMU(惯性测量单元)。实验运行在一台普通的x86-64台式机上,配有8个英特尔处理器和20GB物理内存。使用的操作系统为Ubuntu
20.04, ROS2版本为ROS2 Foxy。
\begin{table}[H]
	\small
	\caption{用于测试的ROS程序信息}
	\label{tb:ros_test}
	\centering
	\begin{tabular}{cccc}
		\hline  
		\textbf{类型} & \textbf{程序} & \textbf{描述} & \textbf{LOC} \\ 
		\hline  
		移动 & nav2\_bt\_navigator & ROS2导航中的BT导航模块 & 6.2K \\
		移动 & nav2\_planner & ROS2导航中的路径规划器模块 & 10.4K \\
		移动 & nav2\_recoveries & ROS2导航中的恢复模块 & 1.2K \\
		移动 & nav2\_controller & ROS2导航中的控制器模块 & 6.3K \\
		定位 & nav2\_amcl & ROS2导航中的定位模块 & 14.4K \\
		定位 & lama\_loc & 可选的定位和映射方法 & 9.5K \\
		定位 & ekf\_loc & 基于卡尔曼滤波的定位方法 & 0.8K \\
		SLAM & rtab-map & 实时RGB-D SLAM方法 & 26.1K \\
		SLAM & slam\_toolbox & 一套用于2D SLAM的工具和功能 & 15.1K \\
		SLAM & cartographer & 实时2D和3D SLAM方法 & 7.8K \\
		\hline
	\end{tabular} 
\end{table}

\subsection{测试成果与漏洞统计}
本测试运行一个通用的第三方分析工具ASan{[}20{]}和本作品来检测运行时的内存bug。每个程序都用相关的机器人任务进行24小时的测试。表格2显示了测试结果。

\begin{table}[H]
	\small
	\caption{机器人程序在ROS2模糊测试结果}
	\centering
	\begin{tabular}{ccccc}
		\hline
		\textbf{程序} & \multicolumn{2}{c}{\textbf{覆盖分支}} & \multicolumn{2}{c}{\textbf{漏洞检测}} \\
		\cline{2-5}
		& \textbf{模糊测试} & \textbf{测试样例} & \textbf{发现的漏洞} & \textbf{确认的漏洞} \\
		\hline
		nav2\_bt\_navigator & 24.3K & 17.2K & 3 & 2 \\
		nav2\_planner & 27.9K & 25.9K & 6 & 5 \\
		nav2\_recoveries & 15.8K & 16.4K & 2 & 2 \\
		nav2\_controller & 37.6K & 25.1K & 3 & 1 \\
		nav2\_amcl & 12.2K & 12.0K & 9 & 7 \\
		lama\_loc & 25.3K & 20.7K & 3 & 0 \\
		ekf\_loc & 16.2K & 5.2K & 1 & 0 \\
		rtab-map & 70.9K & 59.3K & 8 & 1 \\
		slam\_toolbox & 30.9K & 10.7K & 3 & 2 \\
		cartographer & 59.9K & 52.1K & 5 & 0 \\
		\textbf{总计} & \textbf{320.9K} & \textbf{244.5K} & \textbf{43} & \textbf{20} \\
		\hline
	\end{tabular}
\end{table}

\textbf{测试覆盖率。}为了理解本作品的测试覆盖改进,本测试还对每个ROS程序运行了24小时的官方测试用例,并收集了覆盖的代码分支的数量。得益于本作品基于ROS属性的模糊测试方法,与运行测试样例相比,本作品覆盖了测试程序中31\%的代码分支。值得一提的是,在运行官方测试用例时,本作品没有发现任何错误。

\textbf{漏洞检测。}本作品在测试程序中发现了43个真实的漏洞,没有误报。这些漏洞实际上可以使用本作品及其生成的测试用例来重现。本团队已经向相关ROS开发者报告了这些漏洞,其中20个漏洞已经得到了确认和修复,仍在等待对剩余漏洞的反馈。值得注意的是,有五个测试程序的名称包含``nav2'',它们是由ROS社区开发和维护的且广泛用于基于ROS的机器人。因此,这5个程序中已确认的17个bug受到了ROS开发者的高度关注。
\subsection{ROS漏洞实例分析}
本团队还根据发现的43个漏洞的类型进行了分类,并将结果总结在表3中。具体来说,有6个空指针解引用,5个释放后使用错误,3个缓冲区/堆栈溢出错误,11个无效指针访问和18个未捕获异常,如表格3所示。一旦这些错误被特定的输入触发,运行时故障和严重的安全问题就会在运行时发生。具体来说,空指针解引用和未捕获的异常可能导致程序崩溃,从而异常中止机器人任务;缓冲区/堆栈溢出错误、释放后使用错误和无效指针访问会导致机器人的未定义行为,并增加机器人受到恶意攻击的风险。下面将对不同种类的漏洞进行简要介绍其特点及危害。
\begin{table}[H]
	\small
	\caption{发现漏洞的种类}
	\centering
	\begin{tabular}{ccccccc}
		\hline
		\textbf{程序} & \textbf{空指针} & \textbf{释放后使用} & \textbf{溢出} & \textbf{无效指针} & \textbf{异常} & \textbf{总计} \\
		\hline
		nav2\_bt\_navigator & 1 & 0 & 0 & 0 & 2 & 3 \\
		nav2\_planner & 0 & 2 & 1 & 2 & 1 & 6 \\
		nav2\_recoveries & 0 & 2 & 0 & 0 & 0 & 2 \\
		nav2\_controller & 0 & 0 & 1 & 1 & 1 & 3 \\
		nav2\_amcl & 2 & 0 & 0 & 6 & 1 & 9 \\
		lama\_loc & 2 & 0 & 0 & 0 & 1 & 3 \\
		ekf\_loc & 0 & 0 & 0 & 0 & 1 & 1 \\
		rtab-map & 0 & 0 & 1 & 1 & 6 & 8 \\
		slam\_toolbox & 1 & 1 & 0 & 1 & 0 & 3 \\
		cartographer & 0 & 0 & 0 & 0 & 5 & 5 \\
		\textbf{总计} & \textbf{6} & \textbf{5} & \textbf{3} & \textbf{11} & \textbf{18} & \textbf{43} \\
		\hline
	\end{tabular}
\end{table}

本测试检测到的5个释放后使用漏洞是由数据竞争引起的。具体来说,一个内存对象在一个线程中被释放,但这个对象仍然在另一个线程中使用,没有同步。由于并发执行的不确定性,在正常执行中很难发现这些漏洞。一个真实含有释放后使用漏洞的代码如图表1所示

本测试检测到的这18个未捕获的异常中,有8个错误是由被测程序代码中的内部异常引起的,10个错误是由被测ROS程序使用的第三方软件库和ROS核心组件的API调用的外部异常(在c++中属于``运行时错误''类型)引起的。这个特性表明应该在ROS程序中小心地捕获和处理内部和外部异常。例如,在nav2\_bt\_navigator、nav2\_planner、nav2\_controller和nav2\_amcl中,没有捕捉到来自ROS
rclcpp组件的关于负时间间隔的四个外部异常,这可能导致程序崩溃。真实的代码案例如下图表2所示:

在ROS程序初始化过程中出现了11个bug。这些错误中有6个是由于初始化过程中对无效用户数据和不正确参数配置的错误处理引起的;还有5个漏洞是由初始化过程和消息处理之间缺少同步引起的。因此,在测试ROS程序时,应该特别注意初始化过程。

通过手工检查发现漏洞的回溯,本团队发现有7个漏洞位于被测试的ROS程序所使用的第三方库和ROS核心组件中。这些库和组件包括libopencv、fasttps、eigen、tf2和rclcpp。由于在实验中使用Asan仅仅检测被测试的ROS程序,而不是这些库或组件,因此无法找到这些漏洞的准确位置。为了解决这些问题,本作品尝试使用ASan来检测这些库和组件。但在尝试中发现,它们都不能支持ASan。即便如此,本团队在不使用ASan的情况下手动检查源代码并且成功在rclcpp中找到了一个漏洞(堆栈溢出)。目前rclcpp开发人员已经确认并修复了这个漏洞。

7个漏洞是由回调函数的并发性引起的。在ROS程序中,外部事件(如消息到达)随时会发生,并且每种事件均是由回调函数处理。由于事件可以在任何时间发生,它的回调函数可以与其他函数并发执行。因此,由于不正确的同步,回调函数中可能出现并发错误。图6(a)显示了nav2\_planner中的一个示例错误。回调函数footprint.callback可以与函数getFootprint并发执行。在footprint.callback函数中,指针footprint\_与msg在第106行一起被分配,因此由footprint\_所指向的内存会根据智能指针的功能被释放。同时,getFootprint中仍然使用这个智能指针来访问第76行中的footprint\_-\textgreater polygon,因此导致释放后使用的漏洞。{[}35{]}

11个漏洞是由错误处理问题引起的。具体来说,其中6个错误是由于对无效用户数据和错误的配置参数缺少或不正确的安全检查而引入的;另外5个错误是由于正确的安全检查后错误处理不正确而引入的。图6(b)显示了nav2\_controller中的一个示例错误。transformLaserScanToPointCloud函数可以在运行时抛出一个显式异常和一个隐式异常。在第294行只捕获显式异常,但未捕获``运行时错误''类型的隐式异常{[}36{]}。

2个漏洞是由堆栈内存问题引起的。其中一个漏洞是由在新线程中使用局部变量引起的,另一个漏洞是由过多的递归调用引起的。图6(c)显示了rtab-map中的一个示例错误。在CoreWrapper类的构造函数中,主线程的第141行定义了一个局部变量tfDelay,但是这个变量是在一个通过new
std::thread创建的新线程中的第590行被访问的。由于新线程无法访问主线程的堆栈,因此会出现堆栈溢出错误{[}37{]}。

\subsection{与同类技术对比}
本测试通过实验将本作品与两种最先进的机器人程序测试方法Ros2-fuzz{[}10{]}和ASTAA{[}38{]}进行了比较。ROS2
-fuzz是一种基于AFL{[}2{]}的自动化模糊测试方法,用于在ROS2中测试机器人程序。这种方法会对给定ROS节点中特定主题的消息进行变异。由于Ros2-fuzz是开源的,我们从源代码构建它。ASTAA是一种针对机器人程序鲁棒性测试的方法。它在ROS节点之间随机改变消息,并且还可以在运行时丢弃一些消息以模拟通信不稳定。由于ASTAA是闭源的,本团队通过修改本作品实现了一个类似ASTAA的工具,只允许传感器消息的数据突变和随机丢弃消息,而不使用程序反馈。

在实验中,我们在表1中选择了5个名称包含``nav2''的程序,并运行Ros2-fuzz、ASTAAlike工具和本作品对这些程序进行了24小时的机器人导航任务测试。表4显示了比较结果,其中包括覆盖的代码分支和发现的漏洞。
\begin{table}[H]
	\small
	\caption{不同测试方法的漏洞检测结果}
	\centering
	\begin{tabular}{ccccccc}
		\hline
		\multirow{2}{*}{\textbf{程序}} & \multicolumn{2}{c}{\textbf{Ros2-fuzz}} & \multicolumn{2}{c}{\textbf{ASTAA-like}} & \multicolumn{2}{c}{\textbf{本作品}} \\
		\cline{2-7}
		& \textbf{分支} & \textbf{发现漏洞数} & \textbf{分支} & \textbf{发现漏洞数} & \textbf{分支} & \textbf{发现漏洞数} \\
		\hline
		nav2\_bt\_navigator & 1.8K & 0 & 23.7K & 1 & 24.3K & 3 \\
		nav2\_planner & 1.6K & 0 & 26.4K & 2 & 27.9K & 6 \\
		nav2\_recoveries & 4.8K & 0 & 15.1K & 1 & 15.8K & 2 \\
		nav2\_controller & 3.1K & 0 & 35.2K & 3 & 37.6K & 3 \\
		nav2\_amcl & 0.7K & 0 & 11.9K & 2 & 12.2K & 9 \\
		\textbf{总计} & \textbf{12.0K} & \textbf{0} & \textbf{112.3K} & \textbf{9} & \textbf{117.8K} & \textbf{23} \\
		\hline
	\end{tabular}
\end{table}
本作品发现了Ros2-fuzz和ASTAA-like的工具发现的所有9个漏洞,并且它还发现了这些方法遗漏的14个漏洞,且具有更高的代码覆盖率。实际上,Ros2-fuzz和ASTAA仅生成关于ROS节点之间消息的测试用例,因此在模糊测试期间没有涉及处理不同用户数据和配置参数的大量代码。相比之下,本作品从多个维度(包括用户数据、配置参数和传感器消息)生成测试用例,因此本作品覆盖了Ros2-fuzz和ASTAA遗漏的更多代码。此外,本作品使用分布式分支覆盖更有效地指导多个ROS节点的测试用例生成,并使用三种常见模式执行时间突变(ASTAA只考虑其中一种模式,即消息丢弃),以更有效地覆盖有关时间特征的代码。由于这些原因,在实验中,本作品比Ros2-fuzz和类ASTAA-like的工具产生更好的结果。

通过分析被覆盖分支随着测试时间的增长,观察到随着时间的推移,这三种工具覆盖的新代码分支越来越少。这是因为许多代码分支已经被模糊测试期间由早期突变生成的测试用例所覆盖。即便如此,本团队观察到在后面的测试中,由于本项目的多维生成方法、分布式分支覆盖和时间突变策略,本作品覆盖了更多的新代码分支。

本作品也适用于在ROS1中测试机器人程序。因此,本团队使用本作品和ASan来测试了ROS1中三个常见的机器人程序,包括move\_base
{[}39{]}, nav1\_
amcl{[}40{]}和hector\_mapping{[}41{]}。对于机器人导航任务,本团队运行move\_base和nav1\_amcl,然后运行move\_base和hector\_mapping来完成地图建立任务。本团队对每个程序及其相关任务测试24小时。与第IV-A节类似,同时还运行每个程序的官方测试样例24小时,以验证本作品的测试覆盖改进。表5显示了测试结果。
\begin{table}[H]
	\small
	\caption{ROS1中机器人程序测试结果}
	\centering
	\begin{tabular}{cccccc}
		\hline
		\multirow{2}{*}{\textbf{程序}} & \multicolumn{2}{c}{\textbf{覆盖分支}} & \multicolumn{3}{c}{\textbf{发现的漏洞}} \\
		\cline{2-6}
		& \textbf{模糊测试} & \textbf{测试套件} & \textbf{无效指针} & \textbf{异常} & \textbf{总计} \\
		\hline
		move\_base & 36.4K & 31.9K & 0 & 2 & 2 \\
		navl\_amcl & 12.5K & 10.9K & 2 & 2 & 4 \\
		hector\_mapping & 13.1K & 11.5K & 0 & 0 & 0 \\
		\textbf{总计} & \textbf{62.0K} & \textbf{54.3K} & \textbf{2} & \textbf{4} & \textbf{6} \\
		\hline
	\end{tabular}
\end{table}

与运行测试样例相比,本作品在测试程序中覆盖了14\%的代码分支,这得益于我们基于ROS属性的模糊测试方法。值得注意的是,在运行官方测试样例时,我们没有发现任何错误。由于测试覆盖率的提高,本作品发现了6个真正的错误,没有误报,包括2个无效指针访问和4个未捕获的异常。本团队已经向相关ROS开发人员报告了这些报告,但尚未收到任何回应。实际上,这三个测试程序的github存储库已经很长时间没有更新了。