\subsection{测试环境介绍}
本作品在ROS2中测试了10个常见的机器人程序。根据最新版本的测试结果得到下列图表,表\ref{tb:ros_test}显示了这些用于测试的ROS程序的信息(LOC指代码行数)。为了执行机
器人导航(用于移动和定位程序)和地图构建(用于移动和SLAM程序)任务,这些程序在机器人
仿真框架Gazebo11.5$^{[4]}$中的虚拟机器人TurtleBot3 Waffle上执行。本测试使用的虚
拟传感器包括激光雷达,里程计,2D相机和IMU(惯性测量单元)。实验运行在一台x86-64台
式机上,配有32个Intel处理器核心和64GB物理内存。使用的操作系统为Ubuntu 22.04,
ROS2版本为ROS2 Humble。
\begin{table}[H]
	\small
	\caption{用于测试的ROS程序信息}
	\label{tb:ros_test}
	\centering
	\begin{tabular}{cccc}
		\hline  
		\textbf{类型} & \textbf{程序} & \textbf{描述} & \textbf{LOC} \\ 
		\hline  
    导航 & navigation2 & ROS2导航系统 & 25.8K \\
		定位 & nav2\_amcl & ROS2导航中的定位模块 & 14.4K \\
		定位 & lama\_loc & 可选的定位和映射方法 & 9.5K \\
		定位 & ekf\_loc & 基于卡尔曼滤波的定位方法 & 0.8K \\
		建图 & rtab-map & 实时RGB-D SLAM方法 & 26.1K \\
		建图 & slam\_toolbox & 一套用于2D SLAM的工具和功能 & 15.1K \\
    底层 & ros2rclcpp & ROS2底层基础设施 & 177.1K \\
		\hline
	\end{tabular} 
\end{table}

\subsection{测试成果与漏洞统计}
在作品开发过程中,我们对上述项目进行了广泛测试,表格\ref{tab:test_result}显示了
截稿时的漏洞发掘结果。

\begin{table}[H]
	\small
	\caption{机器人程序在ROS2模糊测试结果}
  \label{tab:test_result}
	\centering
	\begin{tabular}{ccccc}
		\hline
		\textbf{程序} & \multicolumn{2}{c}{\textbf{覆盖分支}} & \multicolumn{2}{c}{\textbf{漏洞检测}} \\
		\cline{2-5}
		& \textbf{模糊测试} & \textbf{测试样例} & \textbf{发现的漏洞} & \textbf{受修复漏洞} \\
		\hline
		navigation2 & 37.6K & 36.8K & 12 & 8 \\
		lama\_loc & 20.8K & 20.7K & 2 & 0 \\
		ekf\_loc & 3.3K & 5.4K & 1 & 0 \\
		rtab-map & 70.9K & 59.3K & 3 & 1 \\
		slam\_toolbox & 12K & 10.7K & 3 & 2 \\
		cartographer & 51.9K & 52.1K & 4 & 0  \\
    rclcpp & None & 47.2K & 3 & 0 \\
		\textbf{总计} & None & None & 28 & 11 \\
		\hline
	\end{tabular}
\end{table}

\textbf{测试覆盖率。}本作品初始输入样例(即未经过自动生成的初始测试输入)包含了
官方的测试用例,但基于这部分测试样例并没有发现新的漏洞,对代码覆盖率贡献也较低,
侧面反映了传统测试用例的低效性。本作品能否稳定提高开源程序测试代码覆盖率,部分项
目由于多版本问题覆盖率才略低于官方测试样例。

\textbf{漏洞检测。本作品在测试程序中发现了28个真实的漏洞,并提供了稳定的复现办
法。本团队已经向相关ROS开发者报告了这些漏洞,其中活跃项目(如navigation2)中11个
漏洞已经得到了确认和修复,部分低活跃程序仍在等待回复。其中Navigation2项目中4个危
险漏洞被美国通用漏洞披露库(CVE)收录,这充分证明了本作品检测出的漏洞的价值。}

\subsection{ROS漏洞实例分析}
% 2024-04-03
% author: yjw

下面挑选我们的漏洞测试系统在ROS相关项目中检测出的几个典型漏洞进行分析,它们充分
体现了本系统的价值。在Navigation2 Issue\#4177中,我们指出了一个典型的数组越界漏
洞。

Navigation2 系统中的节点会监听 \texttt{\textbackslash costmap},来更新代价地图信
息,但是缺乏对代价地图信息合法性的检查。如代码段\ref{lst:bug:msg}中的map消
息,width和height字段指出了代价地图的宽高,data字段给出了代价地图的数据。对于正
常消息,$width\times height$应该等于data数组的大小,但这一点对于传输错误或恶意投
放的消息这并不是被严格保证的。

\begin{figure}[H]
\begin{lstlisting}[]
ros2 topic pub /map nav_msgs/msg/OccupancyGrid "
...
info:
  ...
  width: 2
  height: 2
  ...
data: [-1, 0, 1, 1, -1]  " <- would cause buffer overflow
\end{lstlisting}
  \caption{漏洞\#4177恶意消息}
\label{lst:bug:msg}
\end{figure}

Navigation2 自动导航项目并没有对这一点进行检查,而直接用width和height声明存储
data的数组变量。当\texttt{width}和\texttt{height}大于数组\texttt{data}的实际大小
时,会导致程序访问越界(SEGV)。初始化地图时该bug可能触发,见代码段
\ref{lst:bug:segv}第6行;运行时,如果接收到外部更新的地图信息,也可能触发该问
题,见代码段\ref{lst:bug:segv}第24行。

\begin{figure}[H]
\begin{lstlisting}[language=C++]
/* 初始化时可能发生 */
// create the costmap
costmap_ = new unsigned char[size_x_ * size_y_];
for (unsigned int it = 0; it < size_x_ * size_y_; it++) {
  data = map.data[it]; // <- SEGV
  if (data == nav2_util::OCC_GRID_UNKNOWN) {
    costmap_[it] = NO_INFORMATION;
  } else {
    ...
  }
}

/* 过程中也可能发生,比如接收到外部更新的地图信息 */
void CostmapSubscriber::toCostmap2D()
{
  auto msg = std::atomic_load(&costmap_msg_);
	...
  unsigned char * master_array = costmap_->getCharMap();
  unsigned int index = 0;
  for (unsigned int i = 0; i < msg->metadata.size_x; ++i) {
    for (unsigned int j = 0; j < msg->metadata.size_y; ++j) {
      master_array[index] = msg->data[index]; //<-SEGV
      ++index;
    }
  }
}
\end{lstlisting}
  \caption{漏洞\#4177问题代码}
\label{lst:bug:segv}
\end{figure}

在Naviagtion2PR\#3972, PR\#3958, Issue\#3940$^{[13]}$中,我们报告了一个典型的并发
bug,本系统通过插桩延迟技术,找到了Navigation2项目中Costmap结构数据竞争的根本原
因,删除了之前大量不必要的检查。

此bug发生在关闭期间,ROS系统使用Lifecycle状态机机制来管理节点生命周
期,Controller、Planner、 CostmapROS都是具有生命周期的节点,CostmapROS是
Controller和Planner的子节点(独立线程)。当ROS2系统关闭时,Lifecycle状态机发送同
一关闭信号,这三种节点都会直接对该信号进行响应然后执行资源释放程序,此时三者关闭
顺序是不一致且不确定的。

如果子线程CostmapROS节点已经自行释放,Controller或Planner实质失去了对其的控制
权,如果它们仍试图访问CostmapROS,就产生了重复释放(UAF)、资源冲突等一系列并发
问题,如伪代码\ref{lst:bug:issue3972}所示。

\begin{figure}[H]
\begin{lstlisting}[language=C++]
// Planner or Controller:
Controller::on_cleanup(const rclcpp_lifecycle::State &){
	costmap_ros_->cleanup() // <-UAF, costmap already cleanup itself.
}
\end{lstlisting}
  \caption{漏洞\#3972问题代码}
  \label{lst:bug:issue3972}
\end{figure}

对于需要频繁开关的机器人节点,关闭期间内存漏洞可能产生严重后果,但往往被开发者忽
略。在 Navigation2 Issue\#4175,PR\#4180,Pull\#4176,和 RclCpp Issue\#2447
$^{[14]}$ 中,我们和ROS2开发者讨论了ROS2节点退出机制可能导致的一些并发问题。

ROS2惯例是将收到订阅消息、收到服务请求的回调函数交给执行器(Executor)执行,因此
执行器线程拥有该回调函数的函数指针,当该回调函数所属的类已被释放,如伪代码
\ref{lst:bug:issue4176}第5行,执行器仍可能访问该回调函数,造成UAF错误。

\begin{figure}[H]
\begin{lstlisting}[language=C++]
nav2_util::CallbackReturn
AmclNode::on_cleanup(const rclcpp_lifecycle::State & /*state*/){
  ...
  initial_pose_sub_.rest(); // <- UAF
  
  ...
  executor_thread_.reset();
}

\end{lstlisting}
  \caption{漏洞\#4176问题代码}
  \label{lst:bug:issue4176}
\end{figure}



% 下面是原论文的ROS漏洞实例分析
本团队还根据发现的28个漏洞的类型进行了分类,并将结果总结在表
\ref{tab:test_bug_types}中。具体来说,有7个并发类型漏洞(包括数据竞争和少数死
锁),7个释放后使用错误(UAF),3个缓冲区/堆栈溢出错误,3个无效指针访问(SEGV)
和8个未捕获异常。一旦这些错误被特定的输入触发,运行时故障和严重安全问题就会发
生,严重威胁机器人的安全性和可靠性。

\begin{table}[H]
	\small
	\caption{发现漏洞的种类}
  \label{tab:test_bug_types}
	\centering
	\begin{tabular}{ccccccc}
		\hline
		\textbf{程序} & \textbf{数据竞争/死锁} & \textbf{UAF} & \textbf{溢出} & \textbf{SEGV} & \textbf{未捕获异常} & \textbf{总计} \\
		\hline
		navigation2 & 1 & 4 & 3 & 2 & 2 & 12\\
		lama\_loc & 2 & 0 & 0 & 0 & 0 & 3 \\
		ekf\_loc & 0 & 0 & 0 & 0 & 1 & 1 \\
		rtab-map & 0 & 0 & 0 & 1 & 2 & 3 \\
		slam\_toolbox & 2 & 1 & 0 & 0 & 0 & 3 \\
		cartographer & 1 & 0 & 0 & 0 & 3 & 4 \\
    rclcpp & 1 & 2 & 0 & 0 & 0 & 3\\
		\textbf{总计} & \textbf{7} & \textbf{7} & \textbf{3} & \textbf{3} & \textbf{8} & \textbf{28} \\
		\hline
	\end{tabular}
\end{table}

内存安全错误,如UAF、BufferOverflow、SEGV,由于易被攻击者利用和植入恶意程序,往
往更受开发者关注,但尽管如此,超过一半的漏洞威胁仍属于内存安全。由于机器人和自动
驾驶相关项目的应用特殊性,程序异常应给予日志记录而不应直接造成程序异常终止;但由
于复杂开源项目的异常处理管理难度大,常常出现开发者间合作不慎而导致异常未被捕获的
情况,底层开发者可能认为应用层开发者会处理该异常,而顶层开发者却可能也认为底层开
发者已经处理了此异常情况。部分UAF漏洞也可能由数据竞争等并发漏洞引发,部分数据竞
争漏洞大部分时间可能不会影响程序运行,造成隐蔽性极高,开发者确认和修复难度也大;
相反,死锁等阻塞型漏洞,通常由于错误使用ROS2通信机制导致,但容易修复和确认。

% 本测试检测到的5个释放后使用漏洞是由数据竞争引起的。具体来说,一个内存对象在一个线程中被释放,但这个对象仍然在另一个线程中使用,没有同步。由于并发执行的不确定性,在正常执行中很难发现这些漏洞。一个真实含有释放后使用漏洞的代码如图表1所示

% 本测试检测到的这18个未捕获的异常中,有8个错误是由被测程序代码中的内部异常引起的,10个错误是由被测ROS程序使用的第三方软件库和ROS核心组件的API调用的外部异常(在c++中属于``运行时错误''类型)引起的。这个特性表明应该在ROS程序中小心地捕获和处理内部和外部异常。例如,在nav2\_bt\_navigator、nav2\_planner、nav2\_controller和nav2\_amcl中,没有捕捉到来自ROS
% rclcpp组件的关于负时间间隔的四个外部异常,这可能导致程序崩溃。真实的代码案例如下图表2所示:

% 在ROS程序初始化过程中出现了11个bug。这些错误中有6个是由于初始化过程中对无效用户数据和不正确参数配置的错误处理引起的;还有5个漏洞是由初始化过程和消息处理之间缺少同步引起的。因此,在测试ROS程序时,应该特别注意初始化过程。

% 通过手工检查发现漏洞的回溯,本团队发现有7个漏洞位于被测试的ROS程序所使用的第三方库和ROS核心组件中。这些库和组件包括libopencv、fasttps、eigen、tf2和rclcpp。由于在实验中使用Asan仅仅检测被测试的ROS程序,而不是这些库或组件,因此无法找到这些漏洞的准确位置。为了解决这些问题,本作品尝试使用ASan来检测这些库和组件。但在尝试中发现,它们都不能支持ASan。即便如此,本团队在不使用ASan的情况下手动检查源代码并且成功在rclcpp中找到了一个漏洞(堆栈溢出)。目前rclcpp开发人员已经确认并修复了这个漏洞。

% 7个漏洞是由回调函数的并发性引起的。在ROS程序中,外部事件(如消息到达)随时会发生,并且每种事件均是由回调函数处理。由于事件可以在任何时间发生,它的回调函数可以与其他函数并发执行。因此,由于不正确的同步,回调函数中可能出现并发错误。图6(a)显示了nav2\_planner中的一个示例错误。回调函数footprint.callback可以与函数getFootprint并发执行。在footprint.callback函数中,指针footprint\_与msg在第106行一起被分配,因此由footprint\_所指向的内存会根据智能指针的功能被释放。同时,getFootprint中仍然使用这个智能指针来访问第76行中的footprint\_-\textgreater polygon,因此导致释放后使用的漏洞。{[}35{]}

% 11个漏洞是由错误处理问题引起的。具体来说,其中6个错误是由于对无效用户数据和错误的配置参数缺少或不正确的安全检查而引入的;另外5个错误是由于正确的安全检查后错误处理不正确而引入的。图6(b)显示了nav2\_controller中的一个示例错误。transformLaserScanToPointCloud函数可以在运行时抛出一个显式异常和一个隐式异常。在第294行只捕获显式异常,但未捕获``运行时错误''类型的隐式异常{[}36{]}。

% 2个漏洞是由堆栈内存问题引起的。其中一个漏洞是由在新线程中使用局部变量引起的,另
% 一个漏洞是由过多的递归调用引起的。图6(c)显示了rtab-map中的一个示例错误。在
% CoreWrapper类的构造函数中,主线程的第141行定义了一个局部变量tfDelay,但是这个变
% 量是在一个通过new std::thread创建的新线程中的第590行被访问的。由于新线程无法访问
% 主线程的堆栈,因此会出现堆栈溢出错误{[}37{]}。

% \subsection{与同类技术对比}
% 本测试通过实验将本作品与两种最先进的机器人程序测试方法Ros2-fuzz$^{[10]}$和
% ASTAA{[}38{]}进行了比较。ROS2 -fuzz是一种基于AFL$^{[11]}$的自动化模糊测试方法,
% 用于在ROS2中测试机器人程序。这种方法会对给定ROS节点中特定主题的消息进行变异。由
% 于Ros2-fuzz是开源的,我们从源代码构建它。ASTAA是一种针对机器人程序鲁棒性测试的方
% 法。它在ROS节点之间随机改变消息,并且还可以在运行时丢弃一些消息以模拟通信不稳
% 定。由于ASTAA是闭源的,本团队通过修改本作品实现了一个类似ASTAA的工具,只允许传感
% 器消息的数据突变和随机丢弃消息,而不使用程序反馈。

% 在实验中,我们在表1中选择了5个名称包含``nav2''的程序,并运行Ros2-fuzz、ASTAAlike
% 工具和本作品对这些程序进行了24小时的机器人导航任务测试。表4显示了比较结果,其中
% 包括覆盖的代码分支和发现的漏洞。
% \begin{table}[H]
% 	\small
% 	\caption{不同测试方法的漏洞检测结果}
% 	\centering
% 	\begin{tabular}{ccccccc}
% 		\hline
% 		\multirow{2}{*}{\textbf{程序}} & \multicolumn{2}{c}{\textbf{Ros2-fuzz}} & \multicolumn{2}{c}{\textbf{ASTAA-like}} & \multicolumn{2}{c}{\textbf{本作品}} \\
% 		\cline{2-7}
% 		& \textbf{分支} & \textbf{发现漏洞数} & \textbf{分支} & \textbf{发现漏洞数} & \textbf{分支} & \textbf{发现漏洞数} \\
% 		\hline
% 		nav2\_bt\_navigator & 1.8K & 0 & 23.7K & 1 & 24.3K & 3 \\
% 		nav2\_planner & 1.6K & 0 & 26.4K & 2 & 27.9K & 6 \\
% 		nav2\_recoveries & 4.8K & 0 & 15.1K & 1 & 15.8K & 2 \\
% 		nav2\_controller & 3.1K & 0 & 35.2K & 3 & 37.6K & 3 \\
% 		nav2\_amcl & 0.7K & 0 & 11.9K & 2 & 12.2K & 9 \\
% 		\textbf{总计} & \textbf{12.0K} & \textbf{0} & \textbf{112.3K} & \textbf{9} & \textbf{117.8K} & \textbf{23} \\
% 		\hline
% 	\end{tabular}
% \end{table}
% 本作品发现了Ros2-fuzz和ASTAA-like的工具发现的所有9个漏洞,并且它还发现了这些方法遗漏的14个漏洞,且具有更高的代码覆盖率。实际上,Ros2-fuzz和ASTAA仅生成关于ROS节点之间消息的测试用例,因此在模糊测试期间没有涉及处理不同用户数据和配置参数的大量代码。相比之下,本作品从多个维度(包括用户数据、配置参数和传感器消息)生成测试用例,因此本作品覆盖了Ros2-fuzz和ASTAA遗漏的更多代码。此外,本作品使用分布式分支覆盖更有效地指导多个ROS节点的测试用例生成,并使用三种常见模式执行时间突变(ASTAA只考虑其中一种模式,即消息丢弃),以更有效地覆盖有关时间特征的代码。由于这些原因,在实验中,本作品比Ros2-fuzz和类ASTAA-like的工具产生更好的结果。

% 通过分析被覆盖分支随着测试时间的增长,观察到随着时间的推移,这三种工具覆盖的新代码分支越来越少。这是因为许多代码分支已经被模糊测试期间由早期突变生成的测试用例所覆盖。即便如此,本团队观察到在后面的测试中,由于本项目的多维生成方法、分布式分支覆盖和时间突变策略,本作品覆盖了更多的新代码分支。
